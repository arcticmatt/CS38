%%%%%%%%%%%%%%%%%%%%%%%%%%%%%% Preamble
\documentclass{article}
\usepackage{amsmath,amssymb,amsthm,fullpage,enumitem}
\usepackage[a4paper,bindingoffset=0in,left=1in,right=1in,top=1in,
bottom=1in,footskip=0in]{geometry}
\newtheorem*{prop}{Proposition}
%\newcounter{Examplecount}
%\setcounter{Examplecount}{0}
\newenvironment{discussion}{\noindent Discussion.}{}
\pagenumbering{gobble}
\newcommand{\modifyenum}[1]{%
  \end{enumerate}
  \begin{enumerate}[resume,#1]
}
\begin{document}

%%%%%%%%%%%%%%%%%%%%%%%%%%%%%% Problem 1
\section*{Problem 1, CS38 Final, Matt Lim}
The greedy algorithm will be as follows. At each step, we will pick the set
covering the largest number of remaining uncovered items. Now, we will show that
this achieves an approximation ratio of $\frac{e}{e-1}$.

Let $OPT$ be the cardinality of the set of elements covered by a
\textit{maximum k-cover} - that is, the cardinality of the set
with the maximum number of elements of $U$ that can be covered by $k$ of the
subsets. Let $r_i$ be the number of the $OPT$ elements remaining (not yet
covered) after iteration $i$; that is, if $x$ elements have been covered after
iteration $j$, then $r_j = OPT - x$. This means $r_0 = OPT$. Then we claim that
\[ r_i \leq (1 - \frac{1}{k}) \cdot r_{i-1}. \]
The proof for this statement is as follows. We have
that $k$ subsets cover $OPT$ elements. Then, at each step, $k$ subsets cover
all remaining elements (out of the $OPT$ elements, not the universe).
Thus, at each step, \textit{some} subset must cover at least a $\frac{1}{k}$
fraction of those remaining elements out of the $OPT$ elements. That is, at
each step, there
must exist some subset that covers a fraction $\frac{1}{k}$ (at least) of the $r_{i-1}$
elements remaining to be covered. This basically means we can cover at least
$\frac{r_{i-1}}{k}$ uncovered elements at each step $i$. And since our
greedy algorithm picks the set covering the largest number of remaining
uncovered items, we will cover at least $\frac{r_{i-1}}{k}$ uncovered remaining
elements at each step $i$. Then, using this claim, we have that
\[ r_i \leq (1 - \frac{1}{k})^i \cdot OPT \]
\[ r_k \leq (1 - \frac{1}{k})^k \cdot OPT \]
\[ r_k \leq \frac{OPT}{e} \text{ (since } (1-\frac{1}{x})^x \leq \frac{1}{e})\]
So we have that, after $k$ iterations of our algorithm (which means we have
picked $k$ sets) there are less than or equal to $\frac{OPT}{e}$ elements
remaining of the $OPT$ number of elements that are possible to be covered with
$k$ subsets. This means that $c$, the number of elements we have
covered out of the $OPT$ possible, is bounded below as follows:
\[ c \geq OPT - \frac{OPT}{e} \]
\[ c \geq OPT(1 - \frac{1}{e}) \]
\[ c \geq OPT(\frac{e-1}{e}) \]
\[ c(\frac{e}{e-1}) \geq OPT \]
Thus we get our approximation ratio of $\frac{e}{e-1}$.
\newpage

%%%%%%%%%%%%%%%%%%%%%%%%%%%%%% Problem 2
\section*{Problem 2, CS38 Final, Matt Lim}
Here is our algorithm. Note that a ``child edge'' $(u,w)$ of a vertex $u$ is an edge
that goes from $u$ to one of its children $w$.

\vspace{5mm}

\textbf{Maximum-matching-tree}$(\text{tree } G = (V,E))$
\begin{enumerate}
    \item Root the tree $G$ at a node $r$
    \item node $k$
    \item $S = \varnothing$
    \item int $min = \infty$
    \item $OPT_{in}[v] = OPT_{out}[v] = 0$ for all $v \in V$
    \item \textbf{foreach} node $u$ of $G$ in postorder
    \modifyenum{leftmargin=50pt}
    \item \textbf{if} $u$ is not a leaf
    \modifyenum{leftmargin=75pt}
    \item \textbf{foreach} child edge $(u,w)$
    \modifyenum{leftmargin=100pt}
    \item \textbf{if} $OPT_{in}[w] - OPT_{out}[w] < min$
    \modifyenum{leftmargin=125pt}
    \item $min = OPT_{in}[w] - OPT_{out}[w]$
    \item $k = w$
    \modifyenum{leftmargin=75pt}
    \item $OPT_{in}[u] = 1 + OPT_{out}[k] + \sum_{v \in children(u), v \neq k}
    max\{OPT_{out}[v], OPT_{in}[v]\}$
    \item Store the following information in
        $B_{in}[u]$: edge $e = (u,k)$; for all children $v$ of $u$ (except $k$),
        if $OPT_{out}[v] \geq OPT_{in}[v]$ store $B_{out}[v]$ and
        if $OPT_{out}[v] < OPT_{in}[v]$ store $B_{in}[v]$; for $k$, store
        $B_{out}[k]$
    \modifyenum{leftmargin=75pt}
    \item $OPT_{out}[u] =
        \sum_{v \in \text{children($u$)}} max\{OPT_{in}[v], OPT_{out}[v]$\}
    \modifyenum{leftmargin=75pt}
    \item Store the following information in
        $B_{out}[u]$: for all children $v$ of $u$,
        if $OPT_{out}[v] \geq OPT_{in}[v]$ store $B_{out}[v]$ and
        if $OPT_{out}[v] < OPT_{in}[v]$ store $B_{in}[v]$
    \item $min = \infty$
    \modifyenum{leftmargin=25pt}
    \item \textbf{if} $OPT_{in}[r] > OPT_{out}[r]$
    \modifyenum{leftmargin=50pt}
    \item Backtrack starting with $B_{in}[r]$; for
        each $B_{in}[x]$ encountered, add the stored edge $e$ to $S$
    \modifyenum{leftmargin=25pt}
    \item \textbf{else}
    \modifyenum{leftmargin=50pt}
    \item Backtrack starting with $B_{out}[r]$; for each $B_{in}[x]$ encountered,
        add the stored edge $e$ to $S$
    \modifyenum{leftmargin=25pt}
    \item return $S$
\end{enumerate}

\vspace{5mm}
Note that these following sections depend on the fact that $G$ is a tree.

We will now show that our algorithm runs in time $O(|E|)$ operations, where an
operation is as defined in the problem. We will assume that $|E| \geq |V|$
(otherwise just reading in the graph would be greater than $O(|E|)$).
First we iterate through all the nodes in line 5,
of which their are order $|E|$ many of, and set some values. This is clearly
within the time bounds. Then, with the combined for-loops of lines 6 and 8, we
iterate through every edge in the graph and set/compare some variables for each
edge. This is also clearly within the time bounds. Then we can see that in lines 12
and 14, we perform a sum for each non-leaf node. For each node, these sums sum
over its child edges. Thus, overall, each of these sums ends up summing $O(|E|)$
many elements, each of which are of magnitude $O(|E|)$ (since the value of any
$OPT$ value is bounded by the total number of edges). So, the time it takes to do
both of these sums is $O(|E|)$ operations. Next, we have that in lines 13 and 15
we store values from those sums; thus, since we showed the sums are $O(|E|)$, so
are these lines (overall, each set of $B$ values - $B_{in}$s and $B_{out}$s -
stores information about every vertex,
and storing each thing takes constant time since at most we need to compare two
values to store something). Finally, the backtracking at the end just goes through all the $B$ values for
each node and sometimes adds edges to $S$, which is clearly $O(|E|)$ (since this
means we backtrack through the $B$ values of all the nodes, of which there are
order $|E|$ many). Thus this algorithm
runs in time $O(|E|)$ operations.

Now we will show that our algorithm works. We can see that we visit the nodes in
postorder. This is to ensure a node is visited after all its children. For each
node $u$, we consider the optimal score of adding a child edge $e$ of $u$
to the matching and the optimal score if we don't add any of its child
edges to the matching. The former part occurs in lines 8-12. We choose to add
the edge between $u$ and the child vertex $w$ such that $OPT_{in}[w] -
OPT_{out}[w]$ is minimized.
This is to minimize the loss of score when adding that edge to the
matching, which is the same as maximizing the score of the matching.
The latter part occurs in line 14, where we calculate the maximum score of when
we don't include any of $u$'s child edges in the matching. In this way, we
compute the value of a maximum matching for the subtrees rooted at each non-leaf
node (it will be the maximum of $OPT_{in}$ and $OPT_{out}$); for leaf nodes the
value is always $0$ since they have no children.
Thus, we compute the value of a maximum matching for a tree rooted at $r$, which
is just the value of the maximum matching for the whole tree. Now we must
explain why our backtracking works. We have that, for each node $u$, we store
$B_{in}[u]$ and $B_{out}[u]$. $B_{in}[u]$ contains the optimal child edge
$(u,w)$ of $u$ to add to the matching, and the optimal in/out choice for all child vertices of
$u$ (for $w$, the choice is by default ``out'') stored as $B$ values. $B_{out}[u]$ contains the
optimal in/out choice for all child vertices of $u$ stored as $B$ values. Thus, when starting from the
root node $r$, we have the optimal choices (depending on whether or not we
include a child edge of $r$) for all its child vertices contained
in $B_{in}[r]$ and $B_{out}[r]$. Then, we know which one is better to start with from the values
of $OPT_{in}[r]$ and $OPT_{out}[r]$. Once we pick the correct $B$ to start with,
it is clear that we can continue optimally down to the bottom of the
tree; at each level, we have the $B$ values for all its nodes
(obtained from the previous level, unless we are at the top level),
from which we know the optimal choices/$B$
values for the next level. And since, when we are backtracking, we add the edges
stored by $B_{in}$'s, which we know are optimal, we get the maximum matching.
Note that our backtracking never adds an edge that breaks the matching, since
when we include a child edge $(u,w)$, the $B$ value ensures that $w$ does not
have a child edge added to the matching (because we store $B_{out}$ for $w$).
So overall, we maximize the number of edges added to $S$ while still keeping it
a matching.
\newpage

%%%%%%%%%%%%%%%%%%%%%%%%%%%%%% Problem 3
\section*{Problem 3, CS38 Final, Matt Lim}
Here is the pseudo-code of our algorithm to find a maximum sum subsequence. Note
that it does not use divide-and-conquer; \textbf{I asked Professor Umans and he said it
is OK as long as it runs in the desired time bounds.} Let the size of the
list be $n$, and let $L_k$ denote the $k$th element of the list.
Note that we will let the ``value'' of a list be the sum of its elements.

\vspace{5mm}

\textbf{Maximum-sum-subsequence}(List $L$)
\begin{enumerate}
    \item check if $L$ is all non-positive elements by iterating through every
        element. While iterating through, keep track of the maximum element's index.
        If $L$ is all non-positive elements, return index of maximum element.
    \item int $curr = max = 0$
    \item int $begin = end = beginMax = endMax = 1$
    \item \textbf{for} $1 \leq k \leq n$
    {\setlength\itemindent{25pt} \item \textbf{if} $curr + L_k > max$ }
    {\setlength\itemindent{50pt} \item $max = curr + L_k$ }
    {\setlength\itemindent{50pt} \item $curr = curr + L_k$ }
    {\setlength\itemindent{50pt} \item $beginMax = begin$ }
    {\setlength\itemindent{50pt} \item $endMax = k$ }
    {\setlength\itemindent{25pt} \item \textbf{elif} $curr + L_k > 0$ }
    {\setlength\itemindent{50pt} \item $curr = curr + L_k$ }
    {\setlength\itemindent{25pt} \item \textbf{else} }
    {\setlength\itemindent{50pt} \item $curr = 0$ }
    {\setlength\itemindent{50pt} \item $begin = end = k+1$ }
    \item return $beginMax, endMax$
\end{enumerate}

\vspace{5mm}
We will now see why this runs in the desired time bounds. We have that step 1
takes $O(n)$ operations, as it just iterates through the list. Then, the rest of
the algorithm is basically just a for-loop that iterates through all $n$
elements of the list. In the for-loop, we are just adding integers of
magnitude $O(n|\text{max$_i a_i$}|)$, comparing values, and/or setting integer values.
And in each iteration of the loop, the number of these operations that happen is
clearly bounded by a constant. So we have that overall, our algorithm only takes
$O(n)$ operations, which meets the desired bound of $O(n\log n)$ operations.

Now we will explain why this algorithm works. The first line takes care of lists
that have no positive values by simply returning the maximum value. Otherwise,
we do the following. We start from the
beginning and iterate through the whole list. We keep track of the starting and
ending indices of the current subsequence ($begin,end$), the starting and ending
indices of the current maximum subsequence ($beginMax,endMax$), and the values
for the current subsequence and the maximum subsequence ($curr,max$). For every element, we add its
value to $curr$. Let the element we are considering be called $L_k$.
So, we will add that element to $curr$ to get $sum = curr + L_k$. If $sum$ is
greater than $max$, we set the current subsequence to be the maximum and update
some values. If it is not greater than $max$ but greater than $0$, we maintain our current
subsequence and update some values. Else, if $sum$ is less than or equal to $0$, we start a new
subsequence on the next element and update some values. We do this because once the value of a current
subsequence is non-positive, we know that it is better or the same to start a new
subsequence on the next element than to continue on with the current one, as
non-positive values do not help the sum. So overall, our algorithm considers
a subsequence until its value becomes non-positive (keeping track of the
maximum along the way), and when its value does become non-positive, it
considers a new subsequence starting on the next element. Thus, our algorithm
only elongates the current subsequence if it is possible that it is an essential
part of the maximum (that it actually adds to the value), all while keeping track of
the current maximum subsequence
- otherwise, it starts a new one, since extending the previous one cannot help.
In other words, we never make an outright bad move, only moves that are or have
the possibility of being beneficial. And if we make a possibly beneficial move
that hurts the current value, like adding a negative to our subsequence, we have
stored the value and location of the max subsequence before that move.
Thus, overall, our algorithm keeps track of how high a value a subsequence can
have (while keeping track of the position of the maximum value subsequence)
until its value hits zero or less, then starts again on the next element.
So clearly, since we run this algorithm starting at the beginning of the list,
it will not miss finding a maximum sum subsequence.
\newpage

%%%%%%%%%%%%%%%%%%%%%%%%%%%%%% Problem 3
\section*{Problem 4, CS38 Final, Matt Lim}
\begin{description}
    \item[(a)] Given the LP, our constraint matrix $A$ will be as follows. Letting
        $|V| = n$ and $|E| = m$, we have that our constraint matrix $A$ will
        have $m + 2 + m$ rows and $m + n + m$ columns. The first $m$ rows will represent the $m$
        edges (first $m$ rows will be $e_1, \dots, e_m$), the next two rows will be for the middle two
        constraints, and the last $m$ rows will represent the $m$ edges again
        (same order as first $m$ edges). The first $m$ columns will represent
        edges as well (same order as rows), the
        next $n$ columns will represent all the vertices, and the last $m$
        columns will represent all the edges again (same order as rows). Looking at a
        row in the first $m$ rows that represents edge $e_i = (u,v)$ where $u
        \neq s$, there will be a $1$ in the
        column (from the first $m$ columns) that also represents edge $e_i$. Also, there will be a $-1$ in
        the column that represents vertex $u$, and a $1$ in the column that
        represents vertex $v$. All other columns in that row will have a $0$. If
        $u = s$, then the $-1$ in the column that represents $u$ will be
        replaced with a $0$.  Next, the first row in the middle two rows will have
        a $1$ in the column that represents vertex $s$ ($0$s everywhere else),
        and the second row will have a $1$ in the column
        that represents vertex $t$ ($0$s everywhere else). Finally, looking at
        a row in the last $m$ rows that represents edge $e_i = (u,v)$, there
        will be a $1$ in the
        column (from the last $m$ columns) that also represents edge $e_i$ and
        $0$s everywhere else.  We can then see that multiplying this by the
        vector $x_1, \dots, x_m, y_1, \dots, y_n, x_1, \dots, x_m$ and having a vector $c$ that
        constrains the results of the first $m$ rows to be greater than or
        equal to $1$ for the edges that go out from $s$ and $0$ for those that
        don't, the result of the $m+1$ row to be equal to $1$,
        the result of the $m+2$ row to be equal to $0$, and the results of the
        last $m$ rows to be greater than or equal to $0$ (where a ``result'' is the
        number you get by multiplying a row of $A$ by $x$) gives us all our
        constraints. We will now prove that $A$ is unimodular.

        Now we will prove by induction that $A$ is unimodular. To do this we
        will prove that every square submatrix of $A$ has determinant $0$, $1$,
        or $-1$.
        First we will consider the base case. So, consider any submatrix of $A$
        that has dimensions $1 \times 1$. We showed above that every cell of $A$
        is either $0$, $1$, or $-1$. And since the determinant of a $1 \times 1$
        matrix is just the value of the cell, we have that every $1 \times 1$
        submatrix of $A$ has determinant $0$, $1$, or $-1$. Now we will do our inductive
        assumption. So, assume that every $k \times k$ submatrix of
        $A$, where $1 \le k \le i - 1$ has determinant $0$, $1$, or $-1$.
        Now we must show that every $i \times i$ submatrix of
        $A$ has such a determinant. So consider an arbitrary $i \times i$
        submatrix $B$ of $A$.

        Now we will consider four cases. The first case is when a column $c$ in
        $B$ is made of all
        zeroes. Then we have that the determinant of $B$ is zero (expand by
        cofactors along $c$). The second case is
        when a column $c$ in $B$ has one $1$ or one $-1$ and $0$s everywhere else.
        Here we have that the determinant of $B$ is either $0$, $1$,
        or $-1$. We can see this is true if we expand by cofactors along $c$.
        This is because by calculating the determinant this way, it is the same
        as if we calculate the determinant of the $(i - 1) \times (i - 1)$
        matrix we get by crossing out column $c$ and the row with the one $1/-1$ in
        column $c$ and multiplying it by $1$ or $-1$. And by our inductive
        assumption, the determinant of such a $(i - 1) \times (i - 1)$ matrix is
        $0$, $1$, or $-1$. So clearly the determinant of $B$ is $0$, $1$, or
        $-1$.
        Similarly to the first two cases, we
        have a third case in which one row of $B$ is made of all $0$s and a
        fourth case in which one row of $B$ is made of all $0$s and one
        $1$ or $-1$. These cases are proved true in the same way as the first
        two cases, except that instead of expanding along a column we expand
        along a row.
        Note that these first four cases cover every square submatrix that
        includes a row from the $m+1$st row down. This is because
        all those rows only have one $1$. Also note that these first four cases
        cover every square submatrix that includes a column from the first $m$
        columns from $A$ and/or from the last $m$ columns of $A$.
        This is because all those columns only have
        one $1$. This means we have covered all submatrices except for those
        from the top middle $m \times n$ submatrix.

        To cover this part, we get a final fifth case, in which every row of $B$ has
        both one $1$ and one $-1$ and every column has at least 2 non-zero
        values. We will now claim that in this case, it is possible to make a row of
        all zeroes by adding multiples of rows to other rows. We
        can see this is true because of the following reason. Since we have two
        non-zero values in each column (two $1$s, two $-1$s, or
        one $1$ and one $-1$) and exactly one $1$ and one $-1$ in each row,
        we can do the following. Let us add rows to the
        first row. Then, we can just keep on adding/subtracting rows, getting rid of one
        $\pm1$ in a column at least one column at a time. Note that, every time we
        add/subtract a row to/from the first row, the overall value of that row
        stays 0, since we are adding a $1$ and subtracting a $1$ either
        way. Therefore, the first row will eventually become all $0$s,
        since we have two $\pm1$s in each column that we can have cancel
        and we can always keep the value of the first row at $0$.
        For an example, see the following:
        \[
        \begin{bmatrix}
        -1 & 1 & 0 & 0\\
        -1 & 0 & 1 & 0\\
         0 & -1 & 0 & 1\\
         0 & 0 & -1 & 1
        \end{bmatrix} \rightarrow
        \begin{bmatrix}
         0 & 1 & -1 & 0\\
        -1 & 0 & 1 & 0\\
         0 & -1 & 0 & 1\\
         0 & 0 & -1 & 1
        \end{bmatrix} \rightarrow
        \begin{bmatrix}
         0 & 0 & -1 & 1\\
        -1 & 0 & 1 & 0\\
         0 & -1 & 0 & 1\\
         0 & 0 & -1 & 1
        \end{bmatrix} \rightarrow
        \begin{bmatrix}
         0 & 0 & 0 & 0\\
        -1 & 0 & 1 & 0\\
         0 & -1 & 0 & 1\\
         0 & 0 & -1 & 1
        \end{bmatrix}
        \]
        First we subtract the second row, then we add the third row, then we
        subtract the fourth row. We can see that, since we have two $\pm1$s in
        each column and exactly one $1$ and one $-1$ in each row,
        we can just cancel all of them out in the first row.
        So, we have that it is possible to make a row of all zeroes by adding
        multiples of rows to other rows. Note now that it is a theorem of linear
        algebra that the determinant of a square matrix is unchanged if the
        entries in one row are added to those in another row (page 1224 CLRS).
        Then, since we have a row of all zeroes, we can expand along this row
        and get that the determinant of $B$ is $0$. So for this case, the determinant
        is $0$. Now we have that all the cases are covered. Thus, we can
        conclude by induction that every square submatrix of $A$ has determinant
        $0$, $1$, or $-1$, and thus that $A$ is totally unimodular.
    \item[(b)] First, we will show that the dual LP of this given LP is the
        max-flow LP, and that its maximum value is the value of the max flow.
        So, given our LP, we can say the following. We already defined $A^T$.
        Now we have that $y$ is a column vector of height $m + n + m$ that just contains the edge
        variables ($x_e$s) and the vertex variables ($y_v$s) and the edge
        variables again, in that order. We
        have that $c$ is a column vector of height $m + 2 + m$ that, in the
        first $m$ rows, contains a $1$ for
        each edge that goes out from $s$ and $0$ everywhere else, in the $m+1$
        row contains a $1$, in the $m+2$ row contains a $0$, and in the last $m$
        rows contains all $0$s. Finally, we
        have that $b$ is a column vector of height $m + n + m$ that contains all
        the edge capacities in the first $m$ rows, then all $0$s in the next
        $n+m$ rows. Turning this into the
        dual gives us the following. We have that $A$ has $m + n + m$ rows and
        $m + 2 + m$ columns. The first $m$ rows represent edges, and will have a
        $1$ in the column (from the first $m$ columns) that represents the same
        edge as the row. The next $n$ rows represent vertices; each row will
        have a $-1$ for each edge going out from it and a $1$ for each edge
        going into it (only in the first $m$ columns).
        The two exceptions are the rows that represent $s$ and
        $t$; the row that represents $s$ will have an extra value of $1$,
        and the row that represents $t$ will
        have an extra value of $1$ as well. These two values are in the middle two
        columns. The last $m$ rows again
        represent edges; each of these rows will have a $1$ in the column (from
        the last $m$ columns) that represents the same edge as the row. For all
        these rows, all values that were not specified are $0$. So, that
        is $A$. We can have $x$ be a column vector of height $m + 2 + m$. The
        first $m$ rows will indicate how much flow is sent through each edge,
        the next two rows will be $0$, and the last $m$ rows will again indicate
        how much flow is sent through each edge. $b$ is just the same as in the
        dual (but we will add slack variables), as is $c$. Clearly, we maximize
        the right thing, as $c^T x$ is the sum of the flows coming out from the
        source vertex $s$. Then,
        we can then see that constraining $Ax = b$, where we subtract
        non-negative slack variables to the first $m$ rows of $b$, subtract a
        non-negative slack variable from the row that represents vertex $s$, add
        a non-negative slack variable to the row that represents vertex $t$,
        and add non-negative slack
        variables to the last $m$ rows of $b$ gives us all the constraints we
        want. This is because multiplying the first $m$ rows by $x$ and setting
        them equal to the corresponding values in $b$ gives us the
        constraint that the flow in an edge never exceeds the edge's capacity,
        for each edge. Then, multiplying the next $n$ rows by $x$ and setting
        them equal to the corresponding values in $b$ gives us the
        constraint that the flow going into a vertex equals the flow going out of
        a vertex, for each vertex that is not $s$ or $t$. For vertex $s$ we just
        get the constraint that there is a non-negative net outflow,
        and for vertex $t$ we just get the constraint that there is a
        non-negative net inflow. Finally, multiplying the last $m$ rows by $x$
        and setting them equal to the corresponding values in $b$ gives us the
        constraint that all the flows are non-negative. Thus, we have shown that the
        dual of the given LP is the max-flow LP, and that the value the max-flow
        LP gives us is indeed the value of the max flow.

        Now we can use what we just proved in the following way. We have that
        the minimum/optimum value given by this LP equals the maximum value
        given by the max-flow LP (which is just the value of the max-flow).
        This is by the strong duality theorem given in lecture 15, slide 49. We can apply
        this because $A/A^T$ and $c$ are real-valued (both just contain integers).
        $b$ is also clearly real valued, as we have integer
        capacities for each edge. And clearly the dual and primal are nonempty.
        Therefore we have that the optimum value of this LP equals the value of the max-flow.
        Then, by the max-flow min-cut theorem, the value of the max-flow equals the
        capacity of the min-cut. Thus we have that the optimum value of this LP equals
        the value of the min-cut.
\end{description}
\newpage

%%%%%%%%%%%%%%%%%%%%%%%%%%%%%% Problem 3
\end{document}
