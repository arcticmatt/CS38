%%%%%%%%%%%%%%%%%%%%%%%%%%%%%% Preamble
\documentclass{article}
\usepackage{amsmath,amssymb,amsthm,fullpage}
\usepackage[a4paper,bindingoffset=0in,left=1in,right=1in,top=1in,
bottom=1in,footskip=0in]{geometry}
\newtheorem*{prop}{Proposition}
%\newcounter{Examplecount}
%\setcounter{Examplecount}{0}
\newenvironment{discussion}{\noindent Discussion.}{}
\pagenumbering{gobble}
\begin{document}

%%%%%%%%%%%%%%%%%%%%%%%%%%%%%% Problem 1
\section*{Problem 1, CS38 Set 6, Matt Lim}
Given a directed graph $G = (V,E)$, we will define our three matroids as follows
(over the universe $E$). $M_1$
will be the graphic matroid for $G$.
$M_2$ will contain all sets of edges such
that, for every set, every vertex that is part of an edge (an edge in the set)
has at most one edge going out. $M_2$ will also contain all subsets of every such set. $M_3$
will contain all sets of edges such that, for every set, every vertex that is part of an edge
(an edge in the set) has at most one edge going in. $M_3$ will also contain all
subsets of every such set.

We will now prove that each of these matroid is indeed a matroid. Note that we
will not provide a proof for $M_1$ since we proved that graphic matroids are
matroids in problem set 2. So for $M_1$, we just need to give a polynomial-time
procedure that determines whether or not a set $A \subseteq E$ is an independent
set. To do this, it suffices to check if the graph formed by the edges
in $A$ and the vertices included in those edges is acyclic, since if that graph
is acyclic that means the
edges form a forest, and that $A$ is an independent set of $M_1$.
To do this, we can see if there are any strongly connected components with
greater or equal to two vertices in the
graph formed by the edges of $A$ and the vertices in those edges. If there is,
that means that $A$ has a cycle, and that it shouldn't be in the matroid; if
there is not, $A$ is an independent set since it is acyclic. We covered an algorithm for finding
SCCs in class (it's also in the book, page 617)
and it was polynomial time. To be safe we could also check each
vertex to make sure there are no self cycles, which is also polynomial.

So, we will first prove that $M_2$ is a matroid. So,
we will first consider the first axiom, that $M_2$ contains the empty set. This
is trivially true, since the empty set is a subset of every set. Also, having
zero edges in a set meets the requirement that every vertex that is part of an
edge has at most one edge going out, since no vertices will be part of any edges.
Now, on to the second axiom. This is true by construction, since we said that
$M_2$ contains all subsets of the described sets. It is also true because if you
take away an edge, you are not adding any outgoing edges to any vertices. So the
second axiom holds. Finally, we consider the third axiom. So, let $A$ and $B$ be
subsets in $M_2$, with $|B| > |A|$. Now, note that the edges in $A$ result in exactly $|A|$
vertices having one outgoing edge, since every edge added to $A$ adds one
outgoing edge to a new vertex. Similarly, the edges in $B$ result in exactly
$|B|$ vertices having one outgoing edge. Then, since $|B| > |A|$, we have that
$B$ has an outgoing edge $e$ from a vertex $v$ that does not have any outgoing edge
in $A$. So, we can add this edge to $A$, making $A \cup \{e\}$, and have the
resulting set still have at most one outgoing edge from every vertex. Thus $A
\cup \{e\} \in M_2$ and the third axiom holds. Now we will give a
polynomial-time procedure that determines whether or not a set $A \subseteq E$ is an
independent set. This can be done by iterating through all the edges in $A$, and
seeing if any vertices appear at least twice on the left side of the
edges (assuming an edge $(v_1,
v_2)$ means there is an outgoing edge from $v_1$ to $v_2$). If at least one
vertex does, then $A$ is not an independent set in $M_2$. For example, if $A$
includes $(v_1, v_2), (v_1, v_3)$, $v_1$ appears on the left side of edges
twice, and thus has two outgoing edges, so $A$ is not in $M_2$. This is clearly
polynomial since we are just iterating through all the edges in $A$ and
considering two vertices for each edge.

Now we will prove that $M_3$ is a matroid. So,
we will first consider the first axiom, that $M_3$ contains the empty set. This
is trivially true, since the empty set is a subset of every set. Also, having
zero edges in a set meets the requirement that every vertex that is part of an
edge has at most one edge going in, since no vertices will be part of any edges.
Now, on to the second axiom. This is true by construction, since we said that
$M_3$ contains all subsets of the described sets. It is also true because if you
take away an edge, you are not adding any ingoing edges to any vertices. So the
second axiom holds. Finally, we consider the third axiom. So, let $A$ and $B$ be
subsets in $M_3$, with $|B| > |A|$. Now, note that the edges in $A$ result in exactly $|A|$
vertices having one ingoing edge, since every edge added to $A$ adds one
ingoing edge to a new vertex. Similarly, the edges in $B$ result in exactly
$|B|$ vertices having one ingoing edge. Then, since $|B| > |A|$, we have that
$B$ has an ingoing edge $e$ incident to a vertex $v$ that does not have any ingoing edge
in $A$. So, we can add this edge to $A$, making $A \cup \{e\}$, and have the
resulting set still have at most one ingoing edge from every vertex. Thus $A
\cup \{e\} \in M_3$ and the third axiom holds. Now we will give a
polynomial-time procedure that determines whether or not a set $A \subseteq E$ is an
independent set. This can be done by iterating through all the edges in $A$, and
seeing if any vertices appear at least twice on the right side of the
edges (assuming an edge $(v_1,
v_2)$ means there is an ingoing edge into $v_2$). If at least one
vertex does, then $A$ is not an independent set in $M_2$. For example, if $A$
includes $(v_1, v_2), (v_3, v_2)$, $v_2$ appears on the right side of edges
twice, and thus has two ingoing edges, so $A$ is not in $M_3$. This is clearly
polynomial since we are just iterating through all the edges in $A$ and
considering two vertices for each edge.

So we have proved that $M_1$, $M_2$, and $M_3$ are indeed matroids, and given a
polynomial-time procedure for each one that determines whether or not a set
$A \subseteq E$ is an independent set. Now we will
give a reduction from a NP-complete problem to the problem of determining
whether there exists an independent set of cardinality at least $k$ in the
intersection of the three matroids. The NP-complete problem we will reduce from
is Hamilton path in a directed graph. So, let us begin. The reduction function
will take a directed graph $G = (V,E)$ and map it to the three matroids shown
above, as well as an integer $k = |V| - 1$. So, $G = (V,E)$ will get mapped to
$(M_1, M_2, M_3, k)$, where $k = |V| - 1$. Now we must show that yes maps to yes
and no maps to no.

First we will show that yes maps to yes. So, assume that there is a Hamilton
path in $G = (V,E)$. Now we must show that there is an independent set of cardinality $k =
|V| - 1$ or greater in the intersection of $M_1,M_2,M_3$. Let us begin. $M_1$ clearly contains the Hamilton
path because the Hamilton path contains no cycles, and thus is a tree/forest.
$M_2$ contains the Hamilton path because every vertex on the path, by
definition, has exactly one edge going in and exactly one edge going out, except
for the starting and finishing vertex. This means that every vertex along that
path has at most one outgoing edge. By similar logic, $M_3$ contains the
Hamilton path, since every vertex along the Hamilton path has at most one
ingoing edge. Then, since the number of edges along the Hamilton path must be
$|V| - 1$ (by definition of a Hamilton path),
and $k = |V| - 1$, we have that there exists an independent set
in the intersection of $M_1,M_2,M_3$ of cardinality at least $k$, which will be
the edges that make up the Hamilton path.

Now we will show that no maps to no. So, assume that there is not a Hamilton
path in $G = (V,E)$. Now we must show that there is not an independent set of
cardinality $k = |V| - 1$ or greater in the intersection of $M_1, M_2, M_3$. Assume to the
contrary that there exists an independent set in the intersection of
$M_1,M_2,M_3$ that has cardinality at least $k$. This means there is an
independent set in the intersection with the cardinality $k$. Let us consider
the edges in this set.  We have that
these edges don't create a cycle, because they are in $M_1$. We also have that
there are $|V|$ vertices covered, because we have $|V| - 1$ edges, no cycles, and a
maximum of $|V|$ vertices. And since having $|V| - 1$ edges and no cycles hits
at least $|V|$ vertices, we cover $|V|$ vertices. Further, we have that every
vertex covered by the edges in this set has at most one edge going in and one edge going out.
This, along with the fact that we use $|V| - 1$ edges and have no cycles,
means our set of edges is a single directed path, since a path has exactly one edge
going out from and one edge going into every vertex in the path, except for the
first and last vertices. The reasoning behind this is that
enforcing $|V| - 1$ edges with no cycles (means the edges cover all $|V|$
vertices with no cycle)
ensures our set of edges is a single spanning tree if you ignore directions
(not a bunch of unconnected paths), and intersecting $M_2$
and $M_3$ makes sure the tree goes in one direction and doesn't branch, making
it a path. So overall,
we have that there is a set of edges in the intersection of $M_1,M_2,M_3$ that
is a Hamilton path in $G$, since we said our set of edges was a single path that
covered all the vertices. But this is a contradiction. So we have that if there
is not a Hamilton path in $G = (V,E)$, then there cannot be an independent set
of cardinality $k = |V| - 1$, and thus that no maps to no.

So we have that yes maps to yes and no maps to no, and thus we are done.
\newpage

%%%%%%%%%%%%%%%%%%%%%%%%%%%%%% Problem 2
\section*{Problem 2, CS38 Set 6, Matt Lim}
Given a graph $G = (V,E)$ and a parameter $k$, we will do the following to
generate $G' = (V',E')$ and $k'$. We will remove a vertex $v$ such that $deg(v)
> k$, and all of its incident edges. Then we will decrement $k$, so $k' = k -
1$. Then we will remove a vertex $v$ such that $deg(v) > k'$. Then we will
decrement $k'$, so $k'' = k' - 1$. Then we will remove a vertex $v$ such that
$deg(v) > k''$. We will continue on in this manner, stopping when we no longer
can remove a vertex, or when $k^n
= 0$. After this we will remove all vertices of degree zero. This new graph and
$k^n$ will be our $G'$ and $k'$. If this graph does not meet the constraint that
$|E'| \leq k'^2$, so $|E'| > k'^2$, then it follows that $G'$ does not have a
vertex cover of size $k'$. Note that $|V'|
> 2k'^2 \implies |E'| > k'^2$ since $V' \leq 2E'$, so we can just look at the
case when the edge constraint doesn't hold. Now, onto explaining why if $|E'| > k'^2$,
then $G'$ cannot have a vertex cover of size $k'$. This is because our algorithm
made it so that every vertex $v \in G'$ has a degree less than or equal to $k'$. So no
combination of $k'$ vertices will cover all the edges, which number greater than
$k'^2$, because the maximum number of edges possible to cover is $k'^2$. And by the
iff we will prove below, $G$ does not have a $k$ vertex cover since $G'$ doesn't
have a $k'$ one.
So we don't need to run the $O(k^{2k^2} + \text{poly}(n))$ time algorithm for
this case (which needs the constraints we broke anyways), since we already have
determined the absence of a size $k$ vertex cover in $G$. This
procedure is polynomial because every time through, we iterate through at most all the
vertices to check for a degree greater than the current value of $k$. And we can
do this order $|V|$ times since we can remove at most $|V|$ vertices.

Now we will show that $G'$ has a vertex cover of size $k'$ iff $G$ has a vertex
cover of size $k$. First we will show that if $G'$ has a vertex cover of size
$k'$, then $G$ has a vertex cover of size $k$. To do this, it suffices to show
that after every step of the algorithm (every removal of a vertex $v$ with
$deg(v) > k_{current}$), that if the graph has a vertex cover after the
removal, it has one before. So, let us begin. Let $G$ be the
current graph and $k$ the current $k$-value. Let $G'$ be the graph after
removing a vertex $v$ with $deg(v) > k$ and all of its incident edges,
let $k' = k - 1$,  and let $G'$ have a vertex cover of size $k'$. So going from $G$
to $G'$ represents one ``step'' of our algorithm. Now, from $G'$, consider adding back $v$
and all its incident edges. We
have that $v$ covers all those edges. We also have that $k-1$ vertices covered
the rest of the edges which were in $G'$, since $G'$ has a vertex cover.
Thus we have that there is a $k$ size
vertex cover for $G$. Note that removing all the degree zero vertices at the end
doesn't matter because they won't be used in the vertex cover.

Now we will show that if $G$ has a vertex cover of size $k$, then $G'$ has a
vertex cover of size $k'$. To do this, it suffices to show
that after every step of the algorithm (every removal of a vertex $v$ with
$deg(v) > k_{current}$), that if the graph had a vertex cover before the
removal, it has one after. So, let us begin. Let $G$ be the
current graph and $k$ the current $k$-value, and let $G$ have a vertex
cover of size $k$. Let $G'$ be the graph after
removing a vertex $v$ with $deg(v) > k$ and all of its incident edges,
and let $k' = k - 1$. So going from $G$
to $G'$ represents one ``step'' of our algorithm. Now, consider the vertex $v$.
We have that it must be in the vertex cover for $G$, otherwise $deg(v) > k$
number of vertices would be automatically added to the vertex cover. Then we
have that there must be a vertex cover for all the edges that $v$ is not
incident to of size $k - 1$, since there is a vertex cover of size $k$ for the
entire graph that must include $v$. But this is the same as saying $G'$ has a vertex cover of size
$k'$. Note that removing all the degree zero vertices at the end doesn't matter
because they won't be used in the vertex cover.
\newpage

%%%%%%%%%%%%%%%%%%%%%%%%%%%%%% Problem 3
\section*{Problem 3, CS38 Set 6, Matt Lim}
\begin{description}
    \item[(a)]
        We will formulate this problem as a linear program as follows:
        \[ \textbf{maximize } \sum_{p \in P} x_p \]
        \[ \textbf{such that} \]
        \[ \sum_{p \in P, \text{ }p \text{ uses } e} x_p \leq 1 \text{ for each edge $e$} \]
        \[ 0 \leq x_p \leq 1 \text{ for each path $p$} \]
        We can see that this maximizes the total bandwidth utilized, subject to
        the constraint that, for each edge $e$, the sum of the $x_p$ over paths
        that use $e$ is at most 1. This is exactly what the problem wants.
    \item[(b)]
        The primal will be of the form:
        \[ \textbf{maximize } c^Tx \]
        \[ \textbf{such that} \]
        \[ Ax \leq b \]
        \[ x \geq 0 \]
        $x$ is a vector that has the same number of elements as there are
        paths that represents how much bandwidth each path uses (column vector
        with height $|P|$). $A$ has $|E|$ rows and $|P|$ columns, with a one in
        a cell if an edge is used in a path and a zero if an edge is not used in a path.
        $b$ is just a column vector of all ones of height $|E|$. $c^T$ is a
        row vector of all ones of length $|P|$.

        \vspace{5mm}

        The dual will be of the form:
        \[ \textbf{minimize } y^Tb \]
        \[ \textbf{such that} \]
        \[ A^Ty \geq c \]
        \[ y \geq 0 \]
        $c$ and $A^T$ are just the transposes of $c^T$ and $A$ (from the
        primal), respectively, and $b$ is the same as in the primal. We
        can let $y$ be a column vector of height $|E|$ that has an edge
        weight $w(e)$ for each edge $e \in E$. Thus, multiplying a row of $A^T$
        by $y$ gives us the total ``weight'' for the path that row represents,
        since a row of $A^T$ represents which edges are in that path. Thus we
        are constraining the total ``weight'' of each path to be greater than or
        equal to one. Since this puts a constraint on every path, it puts a
        constraint on the minimum length path.
    \item[(c)]
        To implement a separation oracle given a purported solution, the
        solution being the weighting for each edge ($y$), one must simply run Dijkstra's on $G$
        with the purported edge weights
        to find the weight of the shortest path. If this weight is greater than
        or equal to one, we can see that all the constraints are satisfied,
        since this means all paths will have weight greater than or equal to
        one. If this weight is less than one, we have a violated constraint for
        the shortest path. So either we verify all the constraints are satisfied
        or return a violated constraint. And we proved in lecture that
        Dijkstra's is polynomial.
\end{description}
\newpage

%%%%%%%%%%%%%%%%%%%%%%%%%%%%%% Problem 3
\end{document}
