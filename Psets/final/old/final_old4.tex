%%%%%%%%%%%%%%%%%%%%%%%%%%%%%% Preamble
\documentclass{article}
\usepackage{amsmath,amssymb,amsthm,fullpage}
\usepackage[a4paper,bindingoffset=0in,left=1in,right=1in,top=1in,
bottom=1in,footskip=0in]{geometry}
\newtheorem*{prop}{Proposition}
%\newcounter{Examplecount}
%\setcounter{Examplecount}{0}
\newenvironment{discussion}{\noindent Discussion.}{}
\pagenumbering{gobble}
\begin{document}

%%%%%%%%%%%%%%%%%%%%%%%%%%%%%% Problem 1
\section*{Problem 1, CS38 Final, Matt Lim}
The greedy algorithm will be as follows. At each step, we will pick the set
covering the largest number of remaining uncovered items. Now, we will show that
this achieves an approximation ratio of $\frac{e}{e-1}$.

Let $OPT$ be the cardinality of the set of elements covered by a
\textit{maximum k-cover} - that is, the cardinality of the set
with the maximum number of elements of $U$ that can be covered by $k$ of the
subsets. Let $r_i$ be the number of the $OPT$ elements remaining (not yet
covered) after iteration $i$; that is, if $x$ elements have been covered after
iteration $j$, then $r_j = OPT - x$. This means $r_0 = OPT$. Then we claim that
\[ r_i \leq (1 - \frac{1}{k}) \cdot r_{i-1}. \]
The proof for this statement is as follows. We have
that $k$ subsets cover $OPT$ elements. Then, at each step, $k$ subsets cover
all remaining elements (out of the $OPT$ elements, not the universe).
Thus, at each step, at least one of those $k$ subsets must cover at least a $\frac{1}{k}$
fraction of those remaining elements out of the $OPT$ elements. That is, at
each step, there
must exist some subset that covers a fraction $\frac{1}{k}$ (at least) of the $r_{i-1}$
elements remaining to be covered. This basically means we can cover at least
$\frac{r_{i-1}}{k}$ uncovered elements at each step $i$. And since our
greedy algorithm picks the set covering the largest number of remaining
uncovered items, we will cover at least $\frac{r_{i-1}}{k}$ uncovered remaining
elements at each step $i$. Then, using this claim, we have that
\[ r_i \leq (1 - \frac{1}{k})^i \cdot OPT \]
\[ r_k \leq (1 - \frac{1}{k})^k \cdot OPT \]
\[ r_k \leq \frac{OPT}{e} \]
So we have that, after $k$ iterations of our algorithm (which means we have
picked $k$ sets) there are less than or equal to $\frac{OPT}{e}$ elements
remaining of the $OPT$ number of elements that are possible to be covered with
$k$ subsets. This means that $c$, the number of elements we have
covered out of the $OPT$ possible, is bounded below as follows:
\[ c \geq OPT - \frac{OPT}{e} \]
\[ c \geq OPT(1 - \frac{1}{e}) \]
\[ c \geq OPT(\frac{e-1}{e}) \]
\[ c(\frac{e}{e-1}) \geq OPT \]
Thus we get our approximation ratio of $\frac{e}{e-1}$.
\newpage

%%%%%%%%%%%%%%%%%%%%%%%%%%%%%% Problem 2
\section*{Problem 2, CS38 Final, Matt Lim}
Here is our algorithm for finding a maximum matching in a tree $G = (V,E)$ in
time $O(|E|)$ operations. We will consider the edges in order of a breadth-first
search. So the first edge in our table ($e_1$) will be the first edge in a BFS, the
second edge in our table ($e_2$) the second edge in a BFS, etc. Each cell in our table
will contain the information $u_i$ and $v_i$, where the edge $(u_i,v_i)$ is the edge
represented by the cell's column $i$.

\vspace{5mm}
\begin{enumerate}
    \item $OPT[1,i] = 0$ for all $1 \leq i \leq |E|$
    \item $OPT[2,i] = 0$ for all $1 \leq i \leq |E|$
    \item int $i = 1$
    \item while $i \leq |E|$
    {\setlength\itemindent{25pt} \item int $j = i - 1$ }
    {\setlength\itemindent{25pt} \item while $u_j == u_i$ }
    {\setlength\itemindent{50pt} \item $j--$ }
    {\setlength\itemindent{25pt} \item $OPT[1,i] = 1 + OPT[2,j]$ }
    {\setlength\itemindent{25pt} \item $j = i - 1$ }
    {\setlength\itemindent{25pt} \item $OPT[2,i] = OPT[1,j]$ }
    {\setlength\itemindent{25pt} \item while $v_j == v_{j-1} $ }
    {\setlength\itemindent{50pt} \item $j--$ }
    {\setlength\itemindent{50pt} \item $OPT[2,i] = min(OPT[2,i], OPT[1,j])$ }
    {\setlength\itemindent{25pt} \item $i++$ }
\end{enumerate}
\newpage

%%%%%%%%%%%%%%%%%%%%%%%%%%%%%% Problem 3
\section*{Problem 3, CS38 Final, Matt Lim}
\newpage

%%%%%%%%%%%%%%%%%%%%%%%%%%%%%% Problem 3
\section*{Problem 4, CS38 Final, Matt Lim}
\begin{description}
    \item[(a)]
    \item[(b)]
\end{description}
\newpage

%%%%%%%%%%%%%%%%%%%%%%%%%%%%%% Problem 3
\end{document}
