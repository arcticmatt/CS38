%%%%%%%%%%%%%%%%%%%%%%%%%%%%%% Preamble
\documentclass{article}
\usepackage{amsmath,amssymb,amsthm,fullpage}
\usepackage[a4paper,bindingoffset=0in,left=1in,right=1in,top=1in,
bottom=1in,footskip=0in]{geometry}
\newtheorem*{prop}{Proposition}
%\newcounter{Examplecount}
%\setcounter{Examplecount}{0}
\newenvironment{discussion}{\noindent Discussion.}{}
\pagenumbering{gobble}
\begin{document}

%%%%%%%%%%%%%%%%%%%%%%%%%%%%%% Problem 1
\section*{Problem 1, CS38 Set 5, Matt Lim}
\begin{description}
    \item[(a)]
        Note that for $1(a)$ and $1(b)$, we will let the flow and capacity be 0
        for each pair of vertices $u,v$ with $(u,v) \notin E$.
        We will format the vertex-limited flow problem as a linear program in
        the following way. Our original input will consist of the following.
        \[ \textbf{maximize } \sum_{v \in V} f_{sv} - \sum_{v \in V} f_{vs} \]
        \[ \textbf{such that} \]
        \[ f_{uv} \leq c(u,v) \text{ for all $u,v \in V$} \]
        \[ \sum_{v \in V} f_{vu} = \sum_{v \in V} f_{uv} \text{ for each $u \in
        V - \{s,t\}$} \]
        \[ \sum_{v \in V} f_{vu} + \sum_{v \in V} f_{uv} \leq g(v) \text{ for each $u \in
        V$} \]
        \[ f_{uv} \ge 0 \text{ for each $u,v \in V$} \]
        Turning this into standard form then gives us the following.
        \[ \textbf{maximize } \sum_{v \in V} f_{sv} - \sum_{v \in V} f_{vs} \]
        \[ \textbf{such that} \]
        \[ f_{uv} + S_C = c(u,v) \text{ for all $u,v \in V$} \]
        \[ \sum_{v \in V} f_{vu} = \sum_{v \in V} f_{uv} \text{ for each $u \in
        V - \{s,t\}$} \]
        \[ \sum_{v \in V} f_{vu} + \sum_{v \in V} f_{uv} + S_G = g(v) \text{ for each $u \in
        V$} \]
        \[ f_{uv} \ge 0 \text{ for each $u,v \in V$} \]
        \[ S_C, S_G \ge 0 \]
        Now for a short explanation. The first two constraints ensure that the
        flow satisfies the capacity constraint and flow conservation. The next
        constraint satisfies the constraint given in the problem, and the last
        two simply ensure that certain values are non-negative. Basically, all
        the constraints except for the third one ensure we get a valid flow, and
        the third one gives us what the problem wants. Then, with all
        these constraints, we want to maximize the flow from $s$ to $t$.
    \item[(b)]
        Here is how we will format this problem as a linear program. Our
        original input will consist of the following.
        \[ \textbf{maximize } \sum_{e \text{ exiting $s$}} (c(e^*) + 1) \cdot
        f(e) - f(e^*) \] \[ \textbf{such that} \]
        \[ f_{uv} \leq c(u,v) \text{ for all $u,v \in V$} \]
        \[ \sum_{v \in V} f_{vu} = \sum_{v \in V} f_{uv} \text{ for each $u \in
        V - \{s,t\}$} \]
        \[ f_{uv} \ge 0 \text{ for each $u,v \in V$} \]
        Turning this into standard form then gives us the following.
        \[ \textbf{maximize } \sum_{e \text{ exiting $s$}} (c(e^*) + 1) \cdot
        f(e) - f(e^*) \] \[ \textbf{such that} \]
        \[ f_{uv} + S_C = c(u,v) \text{ for all $u,v \in V$} \]
        \[ \sum_{v \in V} f_{vu} = \sum_{v \in V} f_{uv} \text{ for each $u \in
        V - \{s,t\}$} \]
        \[ f_{uv} \ge 0 \text{ for each $u,v \in V$} \]
        \[ S_C \ge 0 \]
        Now for a short explanation. The constraints ensure that the
        flow satisfies the capacity constraint and flow conservation. In other
        words they ensure we get a valid flow. Now we must show that maximizing
        what we maximize gives us a maximum flow and minimizes $f(e^*)$. The
        latter part is easy; maximizing our function clearly minimizes $f(e^*)$,
        since we are subtracting it. Now we must show that maximizing our
        function gives us a max flow. Let the value of the flow we get by
        maximizing our function be $f$, and let the value of the max flow be
        $f_{max}$. $f > f_{max}$ clearly cannot happen because then $f_{max}$ wouldn't be the
        max flow. Now we will show that $f < f_{max}$ cannot happen. So, assume
        to the contrary that $f < f_{max}$. Let $FUNC =  \sum_{e \text{ exiting $s$}} (c(e^*) + 1) \cdot
        f(e) - f(e^*) = A - B$. Now consider $FUNC$ for a flow of value $f_{max}$.
        We have that we have increased $A$ by at least $c(e^*) + 1$, since the
        flow must increase in increments of one. Then we have that $B$ decreases
        by at most $c(e^*)$. So overall $FUNC$ increases by at least one. But
        this is a contradiction because we said we were maximizing $FUNC$. Thus
        we have that $f = f_{max}$, and thus that our linear program will give
        us the maximum flow and minimize $f(e^*)$.
\end{description}
\newpage

%%%%%%%%%%%%%%%%%%%%%%%%%%%%%% Problem 2
\section*{Problem 2, CS38 Set 5, Matt Lim}
\begin{description}
    \item[(a)]
        We will first formulate the Assignment Problem as a problem of finding
        the maximum weight perfect matching in a bipartite graph,
        $G = (L \cup R = V, E)$.  Let $i$ index $L$ and $j$ index $R$, with $v_i
        \in L$ and $v_j' \in R$.
        For all $e \in E$, let $X_e = 1$ if $e$ is selected, and $X_e = 0$
        otherwise. Let $W_e$ be the weight of the edge $e$.
        We will format this problem as a linear program as follows.
        \[ \textbf{maximize } \sum_{e \in E} X_e W_e \]
        \[ \textbf{such that} \]
        \[ \sum_{e \text{ including $v$}} X_e = 1 \text{ for all $v \in V$} \]
        \[ 0 \le X_e \le 1 \text{ for all $e \in E$} \]
        Basically what we are doing is maximizing the total weight of all the
        edges we pick. The first constraint says that each vertex has exactly one
        edge attached to it selected, which means that the edges we select form a perfect
        matching. This is akin to saying each ``agent'' is assigned to exactly one ``task.''
        So overall, we are maximizing the weight of the assignment.

        \vspace{5mm}

        Now we will show that the constraint matrix is totally unimodular. So,
        let us consider the constraint matrix $A$. Our constraint matrix will
        have $2n$ rows and $n^2$ columns. The first $n$ rows will consist of
        $v_1, v_2, \dots, v_i$ (the vertices in $L$) and the next $n$ rows will consist of
        $v_1', v_2', \dots, v_j'$ (the vertices in $R$). The columns will represent
        every possible edge, ordered $(v_1, v_1'), (v_1, v_2'), \dots, (v_2,
        v_1'), (v_2, v_2'), (v_n, v_n')$. We have that each column has
        at most two ones in it. This is because of our first constraint, the one
        that said we only select one edge for each vertex, and because each edge
        contains two vertices. Further, every cell of the matrix is either
        zero or one, because of how we defined our constraint.
        That is, every cell is zero or one because every vertex is included in
        at most one selected edge, and because we either select an edge ($X_e =
        1$) or don't select it ($X_e = 0$). We also have that swapping rows does not affect
        if $A$ is unimodular. This is because it is a linear algebra theorem that says that
        swapping the rows of a square matrix only changes the sign of the
        determinant. It then clearly follows that swapping rows doesn't affect
        unimodularity.

        Now we will prove by induction that $A$ is unimodular.
        First we will consider the base case. So, consider any submatrix of $A$
        that has dimensions $1 \times 1$. We showed above that every cell of $A$
        is either zero or one. And since the determinant of a $1 \times 1$
        matrix is just the value of the cell, we have that every $1 \times 1$
        submatrix of $A$ has determinant zero or one, and thus every $1 \times
        1$ submatrix of $A$ is unimodular. Now we will do our inductive
        assumption. So, assume that every $k \times k$ submatrix of
        $A$, where $1 \le k \le i - 1$ is unimodular. Now we must show that every $i \times i$ submatrix of
        $A$ is unimodular. So consider an arbitrary $(i - 1) \times (i - 1)$
        submatrix $B$ of $A$. We will add a column $c$ and a row $r$ to $B$ to
        form $C$. Now
        we will consider three cases. The first case is when $c$ is made of all
        zeroes. Then we have that the determinant of $C$ is zero (expand by
        cofactors along $c$). So $C$ is unimodular. The second case is when $c$ is made of
        all zeroes and one one. Here we can do the following. We can take the
        row with the one in column $c$ and swap it with the newly added row $r$.
        Then we have that the determinant of this new matrix is either $0$, $1$,
        or $-1$. This is because the determinant of the top $(i - 1) \times (i -
        1)$ submatrix is $0$, $1$, or $-1$, and by expanding by cofactors along
        $c$
        with our switched rows we have that the determinant of our new matrix is
        the same as the determinant of the top $(i - 1) \times (i - 1)$
        submatrix, except with a possible sign switch. The third and final case
        is when $c$ is made up of all zeroes and two ones.
    \item[(b)]
\end{description}
\newpage

%%%%%%%%%%%%%%%%%%%%%%%%%%%%%% Problem 3
\section*{Problem 3, CS38 Set 5, Matt Lim}
\newpage

%%%%%%%%%%%%%%%%%%%%%%%%%%%%%% Problem 3
\end{document}
