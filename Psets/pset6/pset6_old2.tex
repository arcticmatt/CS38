%%%%%%%%%%%%%%%%%%%%%%%%%%%%%% Preamble
\documentclass{article}
\usepackage{amsmath,amssymb,amsthm,fullpage}
\usepackage[a4paper,bindingoffset=0in,left=1in,right=1in,top=1in,
bottom=1in,footskip=0in]{geometry}
\newtheorem*{prop}{Proposition}
%\newcounter{Examplecount}
%\setcounter{Examplecount}{0}
\newenvironment{discussion}{\noindent Discussion.}{}
\pagenumbering{gobble}
\begin{document}

%%%%%%%%%%%%%%%%%%%%%%%%%%%%%% Problem 1
\section*{Problem 1, CS38 Set 6, Matt Lim}
Given a directed graph $G = (V,E)$, we will define our three matroids as follows
(over the universe $E$). $M_1$
will be the graphic matroid for the undirected version of $G$.
$M_2$ will contain all sets of edges such
that, for every set, every vertex that is part of an edge (an edge in the set)
has at most one edge going out. $M_2$ will also contain all subsets of every such set. $M_3$
will contain all sets of edges such that, for every set, every vertex that is part of an edge
(an edge in the set) has at most one edge going in. $M_3$ will also contain all
subsets of every such set.

We will now prove that each of these matroid is indeed a matroid. Note that we
will not provide a proof for $M_1$ since we proved that graphic matroids are
matroids in problem set 2. So for $M_1$, we just need to give a polynomial-time
procedure that determines whether or not a set $A \subseteq E$ is an independent
set. To do this, it suffices to check if the graph formed by the edges
in $A$ and the vertices included in those edges is acyclic, since if that graph
is acyclic that means the
edges form a forest, and that $A$ is an independent set of $M_1$.
To do this, we can use a modified version of DFS (as given in the slides).
The change will be that in the for-loop that visits all adjacent vertices, if we
ever find that
$v.color$ is grey (where the vertices $v$ are the adjacent vertices were are
iterating through), then the graph has a cycle. This is basically what I did for
problem $2(d)$ of Problem Set 2. This works because if we are considering a
vertex $u$, if it has an edge to an adjacent vertex $v$, that means $u$ can
reach $v$. If $v$ is grey, it also means that $v$ can reach $u$, since if $v$ is
grey that means that $v$ and $u$ are in the same DFS tree, with $v$ having
already been discovered. So, we have a cycle if $u$ has an edge to a grey vertex
$v$. And since DFS is polynomial, this modified version is polynomial as well
(proved in problem set 1). The short explanation for this is because we still
iterate through the same number of things and only add constant time operations
in the loops.

So, we will first prove that $M_2$ is a matroid. So,
we will first consider the first axiom, that $M_2$ contains the empty set. This
is trivially true, since the empty set is a subset of every set. Also, having
zero edges in a set meets the requirement that every vertex that is part of an
edge has at most one edge going out, since no vertices will be part of any edges.
Now, on to the second axiom. This is true by construction, since we said that
$M_2$ contains all subsets of the described sets. It is also true because if you
take away an edge, you are not adding any outgoing edges to any vertices. So the
second axiom holds. Finally, we consider the third axiom. So, let $A$ and $B$ be
subsets in $M_2$, with $|B| > |A|$. Then we have that $A$ contains at least one
less edge than $B$ does. Now, note that the edges in $A$ result in exactly $|A|$
vertices having one outgoing edge, since every edge added to $A$ adds one
outgoing edge to a new vertex. Similarly, the edges in $B$ result in exactly
$|B|$ vertices having one outgoing edge. Then, since $|B| > |A|$, we have that
$B$ has an outgoing edge $e$ from a vertex $v$ that does not have any outgoing edge
in $A$. So, we can add this edge to $A$, making $A \cup \{e\}$, and have the
resulting set still have at most one outgoing edge from every vertex. Thus $A
\cup \{e\} \in M_2$ and the third axiom holds. Now we will give a
polynomial-time procedure that determines whether or not a set $A \subseteq E$ is an
independent set. This can be done by iterating through all the edges in $A$, and
seeing if any vertices appear at least twice on the left side of the
edges (assuming an edge $(v_1,
v_2)$ means there is an outgoing edge from $v_1$ to $v_2$). If at least one
vertex does, then $A$ is not an independent set in $M_2$. For example, if $A$
includes $(v_1, v_2), (v_1, v_3)$, $v_1$ appears on the left side of edges
twice, and thus has two outgoing edges, so $A$ is not in $M_2$. This is clearly
polynomial since we are just iterating through all the edges in $A$ and
considering two vertices for each edge.

Now we will prove that $M_3$ is a matroid. So,
we will first consider the first axiom, that $M_3$ contains the empty set. This
is trivially true, since the empty set is a subset of every set. Also, having
zero edges in a set meets the requirement that every vertex that is part of an
edge has at most one edge going in, since no vertices will be part of any edges.
Now, on to the second axiom. This is true by construction, since we said that
$M_3$ contains all subsets of the described sets. It is also true because if you
take away an edge, you are not adding any ingoing edges to any vertices. So the
second axiom holds. Finally, we consider the third axiom. So, let $A$ and $B$ be
subsets in $M_3$, with $|B| > |A|$. Then we have that $A$ contains at least one
less edge than $B$ does. Now, note that the edges in $A$ result in exactly $|A|$
vertices having one ingoing edge, since every edge added to $A$ adds one
ingoing edge to a new vertex. Similarly, the edges in $B$ result in exactly
$|B|$ vertices having one ingoing edge. Then, since $|B| > |A|$, we have that
$B$ has an ingoing edge $e$ from a vertex $v$ that does not have any ingoing edge
in $A$. So, we can add this edge to $A$, making $A \cup \{e\}$, and have the
resulting set still have at most one ingoing edge from every vertex. Thus $A
\cup \{e\} \in M_3$ and the third axiom holds. Now we will give a
polynomial-time procedure that determines whether or not a set $A \subseteq E$ is an
independent set. This can be done by iterating through all the edges in $A$, and
seeing if any vertices appear at least twice on the right side of the
edges (assuming an edge $(v_1,
v_2)$ means there is an ingoing edge into $v_2$). If at least one
vertex does, then $A$ is not an independent set in $M_2$. For example, if $A$
includes $(v_1, v_2), (v_3, v_2)$, $v_2$ appears on the right side of edges
twice, and thus has two ingoing edges, so $A$ is not in $M_3$. This is clearly
polynomial since we are just iterating through all the edges in $A$ and
considering two vertices for each edge.

So we have proved that $M_1$, $M_2$, and $M_3$ are indeed matroids, and given a
polynomial-time procedure for each one that determines whether or not a set
$A \subseteq E$ is an independent set. Now we will
give a reduction from a NP-complete problem to the problem of determining
whether there exists an independent set of cardinality at least $k$ in the
intersection of the three matroids. The NP-complete problem we will reduce from
is Hamilton path in a directed graph. So, let us begin. The reduction function
will take a directed graph $G = (V,E)$ and map it to the three matroids shown
above, as well as an integer $k = |V| - 1$. So, $G = (V,E)$ will get mapped to
$(M_1, M_2, M_3, k)$, where $K = |V| - 1$. Now we must show that yes maps to yes
and no maps to no.

First we will show that yes maps to yes. So, assume that there is a Hamilton
path in $G = (V,E)$. Now we must show that there is an independent set of cardinality $k =
|V| - 1$ or greater in the intersection of $M_1,M_2,M_3$. Let us begin. $M_1$ clearly contains the Hamilton
path because the Hamilton path contains no cycles, and thus is a tree/forest.
$M_2$ contains the Hamilton path because every vertex on the path, by
definition, has exactly one edge going in and exactly one edge going out, except
for the starting and finishing vertex. This means that every vertex along that
path has at most one outgoing edge. By similar logic, $M_3$ contains the
Hamilton path, since every vertex along the Hamilton path has at most one
ingoing edge. Then, since the number of edges along the Hamilton path must be
$|V| - 1$ (by definition of a Hamilton path),
and $k = |V| - 1$, we have that there exists an independent set
in the intersection of $M_1,M_2,M_3$ of cardinality at least $k$, which will be
the edges that make up the Hamilton path.

Now we will show that no maps to no. So, assume that there is not a Hamilton
path in $G = (V,E)$. Now we must show that there is not an independent set of
cardinality $k = |V| - 1$ or greater in the intersection of $M_1, M_2, M_3$. Assume to the
contrary that there exists an independent set in the intersection of
$M_1,M_2,M_3$ that has cardinality at least $k$. This means there is an
independent set in the intersection with the cardinality $k$. Let us consider
the edges in this set.  We have that
these edges don't create a cycle, because they are in $M_1$. We also have that
there are $|V|$ vertices covered, because we have $|V| - 1$ edges, no cycles, and a
maximum of $|V|$ vertices. And since having $|V| - 1$ edges and no cycles hits
at least $|V|$ vertices, we cover $|V|$ vertices. Further, we have that every
vertex covered by the edges in this set has at most one edge going in and one edge going out.
This, along with the fact that we use $|V| - 1$ edges and have no cycles,
means our set of edges is a single path, since a path has exactly one edge
going out from and one edge going into every vertex in the path, except for the
first and last vertices. The reasoning behind this is that
enforcing $|V| - 1$ edges with no cycles ensures our set of edges is a single
path ignoring directions (not a bunch of unconnected paths), and intersecting $M_2$
and $M_3$ makes sure the path goes in one direction and doesn't branch. So overall,
we have that there is a set of edges in the intersection of $M_1,M_2,M_3$ that
is a Hamilton path in $G$, since we said our set of edges was a single path that
covered all the vertices. But this is a contradiction. So we have that if there
is not a Hamilton path in $G = (V,E)$, then there cannot be an independent set
of cardinality $k = |V| - 1$, and thus that no maps to no.

So we have that yes maps to yes and no maps to no, and thus we are done.
\newpage

%%%%%%%%%%%%%%%%%%%%%%%%%%%%%% Problem 2
\section*{Problem 2, CS38 Set 6, Matt Lim}
Given a graph $G = (V,E)$ and a parameter $k$, we will do the following to
generate $G' = (V',E')$ and $k'$.
\newpage

%%%%%%%%%%%%%%%%%%%%%%%%%%%%%% Problem 3
\section*{Problem 3, CS38 Set 6, Matt Lim}
\newpage

%%%%%%%%%%%%%%%%%%%%%%%%%%%%%% Problem 3
\end{document}
