%%%%%%%%%%%%%%%%%%%%%%%%%%%%%% Preamble
\documentclass{article}
\usepackage{amsmath,amssymb,amsthm,fullpage}
\usepackage[a4paper,bindingoffset=0in,left=1in,right=1in,top=1in,
bottom=1in,footskip=0in]{geometry}
\newtheorem*{prop}{Proposition}
%\newcounter{Examplecount}
%\setcounter{Examplecount}{0}
\newenvironment{discussion}{\noindent Discussion.}{}
\pagenumbering{gobble}
\begin{document}

%%%%%%%%%%%%%%%%%%%%%%%%%%%%%% Problem 1
\section*{Problem 1, CS38 Set 5, Matt Lim}
\begin{description}
    \item[(a)]
        Note that for $1(a)$ and $1(b)$, we will let the flow and capacity be 0
        for each pair of vertices $u,v$ with $(u,v) \notin E$.
        We will format the vertex-limited flow problem as a linear program in
        the following way. Our original input will consist of the following.
        \[ \textbf{maximize } \sum_{v \in V} f_{sv} - \sum_{v \in V} f_{vs} \]
        \[ \textbf{such that} \]
        \[ f_{uv} \leq c(u,v) \text{ for all $u,v \in V$} \]
        \[ \sum_{v \in V} f_{vu} = \sum_{v \in V} f_{uv} \text{ for each $u \in
        V - \{s,t\}$} \]
        \[ \sum_{v \in V} f_{vu} + \sum_{v \in V} f_{uv} \leq g(v) \text{ for each $u \in
        V$} \]
        \[ f_{uv} \ge 0 \text{ for each $u,v \in V$} \]
        Turning this into standard form then gives us the following.
        \[ \textbf{maximize } \sum_{v \in V} f_{sv} - \sum_{v \in V} f_{vs} \]
        \[ \textbf{such that} \]
        \[ f_{uv} + S_C = c(u,v) \text{ for all $u,v \in V$} \]
        \[ \sum_{v \in V} f_{vu} = \sum_{v \in V} f_{uv} \text{ for each $u \in
        V - \{s,t\}$} \]
        \[ \sum_{v \in V} f_{vu} + \sum_{v \in V} f_{uv} + S_G = g(v) \text{ for each $u \in
        V$} \]
        \[ f_{uv} \ge 0 \text{ for each $u,v \in V$} \]
        \[ S_C, S_G \ge 0 \]
        Now for a short explanation. The first two constraints ensure that the
        flow satisfies the capacity constraint and flow conservation. The next
        constraint satisfies the constraint given in the problem, and the last
        two simply ensure that certain values are non-negative. Then, with all
        these constraints, we want to maximize the flow from $s$ to $t$.
    \item[(b)]
        Here is how we will format this problem as a linear program. Our
        original input will consist of the following. Note that we will let
        $F_{max}$ be the value of the maximum flow from $s$ to $t$.
        \[ \textbf{minimize } f(e^*) \]
        \[ \textbf{such that} \]
        \[ f_{uv} \leq c(u,v) \text{ for all $u,v \in V$} \]
        \[ \sum_{v \in V} f_{vu} = \sum_{v \in V} f_{uv} \text{ for each $u \in
        V - \{s,t\}$} \]
        \[ \sum_{v \in V} f_{sv} - \sum_{v \in V} f_{vs} = F_{max} \]
        \[ f_{uv} \ge 0 \text{ for each $u,v \in V$} \]
        Turning this into standard form then gives us the following.
        \[ \textbf{minimize } f(e^*) \]
        \[ \textbf{such that} \]
        \[ f_{uv} + S_C = c(u,v) \text{ for all $u,v \in V$} \]
        \[ \sum_{v \in V} f_{vu} = \sum_{v \in V} f_{uv} \text{ for each $u \in
        V - \{s,t\}$} \]
        \[ \sum_{v \in V} f_{sv} - \sum_{v \in V} f_{vs} = F_{max} \]
        \[ f_{uv} \ge 0 \text{ for each $u,v \in V$} \]
        \[ S_C \ge 0 \]
        Now for a short explanation. The first two constraints ensure that the
        flow satisfies the capacity constraint and flow conservation. The next
        constraint makes sure that the flow value is indeed maximum, and the last
        two simply ensure that certain values are non-negative. Then, with all
        these constraints, we want the minimize the flow across edge $e^*$.
\end{description}
\newpage

%%%%%%%%%%%%%%%%%%%%%%%%%%%%%% Problem 2
\section*{Problem 2, CS38 Set 5, Matt Lim}
\newpage

%%%%%%%%%%%%%%%%%%%%%%%%%%%%%% Problem 3
\section*{Problem 3, CS38 Set 5, Matt Lim}
\newpage

%%%%%%%%%%%%%%%%%%%%%%%%%%%%%% Problem 3
\end{document}
