%%%%%%%%%%%%%%%%%%%%%%%%%%%%%% Preamble
\documentclass{article}
\usepackage{amsmath,amssymb,amsthm,fullpage}
\usepackage[a4paper,bindingoffset=0in,left=1in,right=1in,top=1in,
bottom=1in,footskip=0in]{geometry}
\newtheorem*{prop}{Proposition}
%\newcounter{Examplecount}
%\setcounter{Examplecount}{0}
\newenvironment{discussion}{\noindent Discussion.}{}
\pagenumbering{gobble}
\begin{document}

%%%%%%%%%%%%%%%%%%%%%%%%%%%%%% Problem 1
\section*{Problem 1, CS38 Final, Matt Lim}
Let $OPT$ be the set of elements covered by a \textit{maximum k-cover} - that is, the set
with the maximum number of elements of $U$ that can be covered by $k$ of the
subsets. Then, the greedy algorithm will be as follows. At each step, we will pick the set
covering the largest number of remaining uncovered items of the set $OPT$. Now, we will show that
this achieves an approximation ratio of $\frac{e}{e-1}$.

Let $r_i$ be the number of the $OPT$ elements remaining after
iteration $i$. This means $r_0 = |OPT|$. Then we claim that
\[ r_i \leq (1 - \frac{1}{k}) \cdot r_{i-1}. \]
The proof for this statement is as follows. We have
that $k$ subsets cover $OPT$. This means that at each step, $k$ subsets
cover the elements in $OPT$ we have not yet covered (the remaining ones).
Thus, at each step, one of those $k$ subsets must cover at least a $\frac{1}{k}$
fraction of those remaining elements. Then, using this claim, we have that
\[ r_i \leq (1 - \frac{1}{k})^i \cdot |OPT| \]
\[ r_k \leq (1 - \frac{1}{k})^k \cdot |OPT| \]
\[ r_k \leq \frac{|OPT|}{e} \]
So we have that, after $k$ iterations of our algorithm (which means we have
picked $k$ sets) there are less than or equal to $\frac{|OPT|}{e}$ elements
remaining of set $OPT$. This means that $c$, the number of elements we have
covered in $OPT$, is bounded below as follows:
\[ c \geq |OPT| - \frac{|OPT|}{e} \]
\[ c \geq |OPT|(1 - \frac{1}{e}) \]
\[ c \geq |OPT|(\frac{e-1}{e}) \]
\[ c(\frac{e}{e-1}) \geq |OPT| \]
Thus we get our approximation ratio of $\frac{e}{e-1}$.
\newpage

%%%%%%%%%%%%%%%%%%%%%%%%%%%%%% Problem 2
\section*{Problem 2, CS38 Final, Matt Lim}
\newpage

%%%%%%%%%%%%%%%%%%%%%%%%%%%%%% Problem 3
\section*{Problem 3, CS38 Final, Matt Lim}
\newpage

%%%%%%%%%%%%%%%%%%%%%%%%%%%%%% Problem 3
\section*{Problem 4, CS38 Final, Matt Lim}
\begin{description}
    \item[(a)]
    \item[(b)]
\end{description}
\newpage

%%%%%%%%%%%%%%%%%%%%%%%%%%%%%% Problem 3
\end{document}
