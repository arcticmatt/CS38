%%%%%%%%%%%%%%%%%%%%%%%%%%%%%% Preamble
\documentclass{article}
\usepackage{amsmath,amssymb,amsthm,fullpage}
\usepackage[a4paper,bindingoffset=0in,left=1in,right=1in,top=1in,
bottom=1in,footskip=0in]{geometry}
\newtheorem*{prop}{Proposition}
%\newcounter{Examplecount}
%\setcounter{Examplecount}{0}
\newenvironment{discussion}{\noindent Discussion.}{}
\pagenumbering{gobble}
\begin{document}

%%%%%%%%%%%%%%%%%%%%%%%%%%%%%% Problem 1
\section*{Problem 1, CS38 Midterm, Matt Lim}
\begin{description}
    \item[(a)]
        See attached sheet.
    \item[(b)]
        We will assume the graph has no self-loops. We will also assume that
        there are no repeats of edges for convenience, since adding
        them or removing them does not matter for detecting triangles.
        Here is pseudo-code for our algorithm. Let $A$ be the adjacency matrix
        of the graph.
        \begin{enumerate}
            \item
                We begin by squaring $A$ to get the matrix $A^2$. This is
                $O(n^{\log _2 7})$, which is less than $O(n^3)$.
            \item
                We then iterate down the diagonal of $A^2$. For each column/row
                $i$ that has a value $v$ greater than or equal to $2$ on the
                diagonal, we will label the corresponding row and column on the
                matrix $A$ with $v$. We will label every other column and row
                with $0$. This label corresponds to how many edges are going
                into a given node (so if a row is labeled with $5$,  there are
                $5$ edges going into it). And given our initial assumptions, it
                also represents how many vertices a given node is connected to.
                This takes time $O(n)$. Note that from here on, we
                will be working with the matrix $A$.
            \item Then, for each row with a value greater than or equal to two, we will iterate
                through its columns. First, we will reset its value to zero.
                Then, for each of its columns (with an index not equal to
                the current row index) with a nonzero value in it, we will
                add one to the row's value. After we have finished reassigning
                the row values, we will copy those values over to the
                corresponding columns. We will then sort the rows
                again in ascending order. This is $O(n^2\log n)$.
            \item
                We will iterate through the rows in ascending order. For each
                row $R$ with a value greater than or equal to $1$, we will iterate
                through its columns. Let $i$ be the index of $R$. If we encounter a
                nonzero value in a column
                that a) is labeled with a nonzero value, b) is not marked by $X$
                (see the end of this item), and c) whose
                index is not $i$, we will store its index in a list $L$. Then, having
                iterated through each column of $R$, we will do the following.
                For every pair of indices
                $j_1$ and $j_2$ in $L$, we will check the $(j_1,j_2)$ position of
                the matrix. If there is a one at that position, then our graph
                has a triangle. If we do this for every pair of indices and we
                find no triangle, then we will mark column $i$ with an $X$.
                \begin{itemize}
                    \item We claim that this step runs in $O(n^2)$. To see why,
                        consider the following argument. When iterating through
                        any column, we will never generate a list with more than
                        $2$ items. This is because we go through in ascending
                        order and because we only add items from columns that
                        are labeled with a nonzero value. It is also because we
                        $X$ out columns once we discover that their
                        corresponding node is not in a triangle.

                        Basically, we must always have
                        at least one node of nonzero value that is
                        attached to at most two nodes of nonzero values that are
                        not crossed out. That is, we must always have at least
                        one node with a value of at most two and at least
                        one. The
                        only other cases are when there is no such node,
                        in which case we are done iterating through the rows, or when the graph is
                        infinite, which we do not have.

                        Since we never generate a list with more than $2$ items,
                        then going through each row only takes $O(n)$. And since
                        there are $n$ rows, we get time
                        $O(n^2)$ for this step.
                \end{itemize}
        \end{enumerate}

        If we follow through the steps, we can see that this algorithm runs in
        the desired time of $O(n^{\alpha})$, for some $\alpha < 3$.

        So we have given the algorithm and given time bounds. Now we will
        explain why this algorithm works. The first three steps label each row
        and column with a number. Let us look at row $R$ that represents node
        $v$ with number $l$. The number $l$ corresponds to how many neighbors
        $v$ has that are connected to two or more nodes. This number is
        important because if a node does not have at least two neighbors
        that are connected to two or more nodes, then it cannot be in a
        triangle. We will call nodes that have at least two neighbors that
        themselves are connected to two or more nodes ``number'' nodes.
        After labeling the rows and columns, we then sort the rows in
        ascending order. This is so that we go through the rows with the lowest
        label first. It is in step 4 that we iterate through the rows.
        For each number node/row, we generate a list of neighbor nodes who are
        also number nodes. Note that we exclude crossed out columns/nodes from
        this list. We only add number nodes because they are the only nodes that
        can be in a triangle. Then, for each pair in the list, we check to see if
        their is an edge between them. If so, we have found a triangle. Overall,
        what this step does is check every possible relevant triple of nodes.
        Thus, it is clear that our algorithm computes the right answer. Note
        that after we determine that a node does not contain a
        triangle, we mark its corresponding column with an $X$. This is so that
        when we are looking at neighbors of this node, we don't consider it
        anymore or put it in the list, since we already know it is not in a
        triangle.
        \end{description}
\newpage

%%%%%%%%%%%%%%%%%%%%%%%%%%%%%% Problem 2
\section*{Problem 2, CS38 Midterm, Matt Lim}
\begin{description}
    \item[(a)]
        We will assume that every element $e \in E$ is mentioned in some subset
        of $I$. This means that for every element $e \in E$, the set $\{e\}$
        is in $I$ (by the second axiom of matroids). Now we will show that the set system on
        $E' = E \backslash \{e\}$ defined by
        \[ I' = \{A : (A \cup \{e\}) \in I\} \]
        is a matroid.

        First we will show that axiom one holds. For $\varnothing \in I'$ to be
        true, then $\varnothing \cup \{e\} = \{e\} \in I$ must be true. And this
        indeed is true by the assumption we made. So axiom one holds.

        Now we will show that the second axiom holds. Let us assume that $A \in
        I'$. Now we must show that every subset $B \subseteq A$ is also in $I'$.
        So, since $A \in I'$, we have that $A' = A \cup \{e\} \in I$. This means that
        every subset $B' \subseteq A'$ is also in $I$, since $I$ is a matroid.
        Then it follows that every subset of $A$ is in $I$. It also follows that
        every subset of $A$ unioned with $\{e\}$ is in $I$. This means that
        every subset of $A$ is in $I'$. So the second axiom holds.

        Finally, we will show that the third axiom holds. So, let $A$ and $B$ be
        subsets in $I'$ with $|B| > |A|$. We must show that there exists $x \in
        B \backslash A$ such that $A \cup \{x\}$ is in $I'$. Let us consider
        $A' = A \cup \{e\} \in I$ and $B' = B \cup \{e\} \in I$. Then we have
        that there exists $x \in B' \backslash A'$ such that $A' \cup \{x\}$ is
        in $I$, where $x \neq e$. Then it follows that we can union this $x$
        with our original $A$ to make $A \cup \{x\}$. Then we can say that $A
        \cup \{x\}$ is in $I'$, since $A \cup \{x\} = (A' \cup \{x\}) \backslash
        \{e\}$. Thus all three axioms hold and we have that $I'$ is a matroid.
    \item[(b)]
        Note that for this problem, we will assume that $F$ does not
        contain a cycle. We will first argue that the subsets of $G$'s edges that, together with
        $F$, form spanning trees, constitute the bases of a matroid $I$. So, let
        these subsets be the bases of $I$ (they have the same cardinality
        because all spanning trees have $|V| - 1$ edges and they are clearly
        independent sets), and let $I$ contain all subsets of them. We want to
        show that this $I$ is a matroid with the desired bases.

        Let $F$ have $n$ edges, $f_1, f_2,\dots, f_n$. Now consider the matroid $I_n$, whose bases are
        the spanning trees of $G$ that contain $F$. We have that all subsets of
        these bases are acyclic, since they are subsets of spanning trees.
        Clearly $I_n$ is indeed a matroid, since it is a graphic matroid. Now,
        consider the matroid $I_{n-1} = \{A : (A \cup \{f_1\}) \in I_n\}$ with
        universe $E' = E \backslash \{f_1\}$. We
        have from part (a) that this a matroid. Then we can continue on in this
        fashion (forming $I_{n-2} = \{A : (A \cup \{f_2\}) \in I_{n-1}\}$, etc),
        continuing to remove edges of $F$,
        until we get the matroid $I$, whose bases are the subsets of $G$'s edges
        that, together with $F$, form spanning trees. The universe of this
        matroid is clearly $E_I = E \backslash F$. Thus we can conclude by
        part (a) that $I$ is indeed a matroid.


        \vspace{8mm}
        Now we will use $I$ to find the desired minimum cost tree.

         Here is a way to formulate this minimum cost spanning tree problem as the
        problem of finding a maximum weight independent set in $I$. We will
        give nonnegative integer
        weights for each element of the universe $E \backslash F$, which will correspond to the
        edge weights of $G$. That is, each element in the matroid will
        represent an edge in the graph, and will be given a positive integer weight.
        However, these edge weights will not be the
        same as the edge weights of the graph. Instead, we will ``flip'' the
        weights. That is, we will take the maximum edge weight from the graph,
        add one to it,
        and subtract the weight of each edge to find the new weight of each
        edge. So, if the max edge weight is $w_{max}$, and we have an edge with
        weight $w$ in the graph, the corresponding weight in the matroid is $w_{matroid edge} =
        w_{max} + 1 - w$. We can rewrite this as $w_{matroid edge} = w_0 - w$, where
        $w_0 = w_{max} + 1$.
        Then we have that all weights are positive,
        and that finding the maximum weight independent set in this
        matroid will give the minimum cost set of edges that, together with $F$,
        forms a spanning tree.
        To see a short proof of this, consider the
        following. Each maximal independent subset $A \in I$ corresponds to a set of
        $|V| - 1 - |F|$ edges. Then we have the following (where $w'(e)$
        represents the weight of the edge $e$ in the matroid, $w(e)$ represents the
        weight of the edge $e$ in the graph, $w'(A)$ represents the sum of the
        matroid edge weights, and $w(A)$ represents the sum of the graph edge
        weights):
        \[ w'(A) = \sum_{e \in A} w'(e) \]
        \[ = \sum_{e \in A} (w_0 - w(e)) \]
        \[ = (|V| - 1 - |F|)w_0 - \sum_{e \in A} w(e) \]
        \[ = (|V| - 1 - |F|)w_0 - w(A) \]
        Clearly, maximizing $w'(A)$ must minimize $w(A)$.

        Now we will prove time bounds and correctness of the greedy algorithm.
        Assigning weights to all the edges is $O(m)$. Then, we proved in set
        number 2 problem number 2b that the greedy algorithm to find the maximum weight
        independent set in a matroid is $O(m \log m)$, where $m$ is the number
        of edges in $G$. In that same problem we also proved correctness. Since
        we have that our greedy algorithm is correct, and given the way we
        defined our edge weights in $I$, it follows that finding a
        maximal independent subset $A$ of $I$ is the same as finding the minimum
        weight subset of $G$'s edges that form a spanning tree with $F$. Then we
        can just add $F$'s edges to our found subset and obtain the minimum cost
        spanning tree containing $F$. Thus our proof is complete.
\end{description}
\newpage

%%%%%%%%%%%%%%%%%%%%%%%%%%%%%% Problem 3
\section*{Problem 3, CS38 Midterm, Matt Lim}
\begin{description}
    \item[(a)]
        Here is our algorithm.
        \begin{enumerate}
            \item First we will sort each list. This is $O(n\log n)$.
            \item We will then make $n$ new lists. Let $A = a_1, \dots, a_n$,
                $B = b_1, \dots, b_n$, and $C = c_1, \dots, c_n$ be the newly sorted
                lists. Then we will have $L_1 = a_1 + b_1, a_1 + b_2, \dots, a_1
                + b_n$, $L_2 = a_2 + b_1, a_2 + b_2, \dots, a_2 + b_n$, etc.
                etc. up until $L_n = a_n + b_1, a_n + b_2, \dots, a_n + b_n$.
                This is $O(n^2)$.
            \item Note that all the lists we made in the previous step are
                sorted. Thus we can do the following.
                We will merge all our lists into one big sorted list. We
                will call this list $L$. We can see that $L$ will have order
                $n^2$ elements. Thus this step is $O(n^2)$.
            \item Then, we will iterate through list $C$. For each element $c_i
                \in C$, we will binary search list $L$ for $-c_i$. If this
                search succeeds for any $c_i$, then there exist $i,j,k$ such that
                $a_i + b_j + c_k = 0$. This is time $O(n\log n^2) =
                O(n\log n)$.
        \end{enumerate}
        Observe that the dominating time bound above is $O(n^2)$. Thus we have
        that our algorithm runs in time $O(n^2)$.
        So we have given the algorithm and shown the time bound for it. Now we
        will show correctness. We have that our list $L$ contains every possible
        sum of one element from $A$ and one element from $B$. In other words,
        $L$ contains every possible $a_i + b_j$. Therefore, we need only to know
        if any element from $L$ plus any element from $C$ sums to $0$. That is,
        each must check if there exists $i,j$ such that $l_i + c_j = 0$, $l_i
        \in L$. Our algorithm does exactly this, checking if if there exists an
        $l_i = -c_j$ for every $c_j \in C$. Thus it follows that our algorithm
        works correctly.
    \item[(b)]
        Note that for this problem we will assume that $M > n$. Now here is our
        algorithm.
        \begin{enumerate}
            \item Let $A = a_1, \dots, a_n$,
                $B = b_1, \dots, b_n$, and $C = c_1, \dots, c_n$ be our lists.
                We will begin by forming two polynomials:
                \[ P_1(X) = \sum_{i \in A} X^{i+M} \]
                \[ P_2(X) = \sum_{i \in B} X^{i+M} \]
                Shifting the degree ensures we have no negative degrees. Forming
                these polynomials is time $O(M)$.
            \item We will then perform fast polynomial multiplication, and form a new
                polynomial:
                \[ P(X) = P_1(X) \cdot P_2(X) \]
                This is $O(M \log M)$ using fast polynomial multiplication as
                seen in lecture, since the largest degree possible is $2M$
                and the smallest possible degree is $0$. This is because
                all of the integers lie in the range $[-M, M]$ and because of
                how we shifted the degrees in the step above.
            \item We will then iterate through $P(X)$ and form a list $L$ of all
                the degrees/powers, with each one shifted down by $2M$. This
                is because we originally shifted all the powers up by $M$.
                We will assume that $P(X)$ has its terms
                sorted in order of ascending degree (e.g. $x + x^2 + x^4 + x^7
                \dots$). Then we have that $L$ is sorted in ascending order.
                Making this list is $O(M)$, since our polynomial has maximum
                degree $2M$ and minimum degree $0$, and we are assuming that
                $M > n$.
            \item Then, we will iterate through list $C$. For each element $c_i
                \in C$, we will binary search list $L$ for $-c_i$. If this
                search succeeds for any $c_i$, then there exist $i,j,k$ such that
                $a_i + b_j + c_k = 0$. This is time $O(M \log M)$, since we are
                assuming that $M>n$.
        \end{enumerate}
        Observe that the dominating time bound above is $O(M \log M)$. Thus we have
        that our algorithm runs in time $O(M \log M)$.
        So we have given the algorithm and shown the time bound for it. Now we
        will show correctness. Consider the polynomial $P(X)$ as seen in item 2.
        Let $D_P$ be the list of the degrees of the terms of $P(X)$. For
        example, for $P(X) = x + x^7$, $D_P = 1, 7$. Then we have that $D_P$
        contains every possible sum of one element from $A$ and one element from
        $B$, except with each sum shifted up by $2M$. But then we can just shift
        each number in the list down by $2M$ so that it actually contains every
        possible sum, which is exactly what we do in our algorithm. So let $D_P$
        be this newly shifted down list.
        Then since $L = D_P$, we have that $L$ contains every possible $a_i
        + b_j$. Therefore, we need only to know
        if any element from $L$ plus any element from $C$ sums to $0$. That is,
        each must check if there exists $i,j$ such that $l_i + c_j = 0$, $l_i
        \in L$. Our algorithm does exactly this, checking if if there exists an
        $l_i = -c_j$ for every $c_j \in C$. Thus it follows that our algorithm
        works correctly.

\end{description}
\newpage

%%%%%%%%%%%%%%%%%%%%%%%%%%%%%% Problem 3
\section*{Problem 4, CS38 Midterm, Matt Lim}
We will use the term ``weight of a triangle'' to indicate the sum of all three
of its edges. Now, here is pseudo-code for our algorithm.

\vspace{5mm}
\noindent \textbf{Lightest-Triangle(P: set of n points in the plane)}
\begin{enumerate}
    \item sort by x coordinate and split equally into \textbf{L} and \textbf{R} subsets
    \item $(a,b,c)$ = \textbf{Lightest-Triangle(L)}
    \item $(d,e,f)$ = \textbf{Lightest-Triangle(R)}
    \item $D_1 = d(a,b) + d(b,c) + d(c,a)$
    \item $D_2 = d(d,e) + d(e,f) + d(f,d)$
    \item $D = min(D_1, D_2)$
    \item scan \textbf{P} by $x$ coordinate to find \textbf{M}: points within $D/2$ of midline
    \item sort \textbf{M} by $y$ coordinate
    \item Starting at the point with the highest y coordinate (call it $v$),
        compute lightest triangle containing $v$ using all the points within
        $24$ positions ahead in the sorted list. Then iterate through the sorted
        list \textbf{M} and do this for every other point. Call the lightest triangle found
        by this iteration $(g,h,i)$.
    \item return lightest among $(a,b,c)$, $(d,e,f)$, and $(g,h,i)$.
\end{enumerate}

\vspace{8mm}
We will now prove time bounds for this theorem. We have that the time is
\[ T(n) = 2T(n/2) + \text{ time for middle } + O(n \log n) \]
where the $O(n \log n)$ is the time it takes to perform the splits and sorts.
Our goal is to show that the time for middle is $O(n \log n)$. To do this,
we need only to look at our algorithm. For each point in the middle area, we
compare at most $\sum_{i=1}^{24} i$ triangles. This is a constant number. And
since the number of points in the middle area is at most $n$, we have that the
middle area takes linear time. So then we get that
\[ T(n) = 2 \cdot T(n/2) + O(n \log n) \]
Then by the proof shown in class (lecture 7, slide 42) we have that
\[ T(n) \leq c \cdot n \cdot \log^2 n \]
So we have that our algorithm runs in time $O(n \log^2 n)$.

\vspace{8mm}
Now we will prove correctness. To do this, it suffices to show that we compute
triangles across the middle correctly. So, let us consider the middle area; that is, the
area composed within distance $D/2$ of the midline (making the total area width
$D$). We only have to consider this area because points outside this area will
have a distance $>D/2$ with points across the midline, which by the triangle
inequality means that any triangle formed by such points will have a weight
$>D$. So, we continue.
We will divide this area up into $D/4 \times D/4$ boxes. Now we can make
the following observations. First, we have that no $3$ points lie in the same
box. If so, then they would form a triangle of weight less than $D$, which is
impossible due to how we set things up. Next, we have that if $2$ points are
within $\ge 24$ positions of each other in list sorted by $y$ coordinate, then
they must be separated by $\ge 2$ rows. Then we have that their distance is $>
D/2$. Then by the triangle inequality, any triangle containing these two points
must have a weight $>D$. So we don't need to consider these points. In other
words, the only possible points that we must consider when going down the sorted
list are the points within 24 positions of each other. And since we consider
points in order of highest to lowest $y$ coordinate, we only need to consider
triangles that can be formed with points within 24 positions ahead of each
point. So we consider every relevant triangle across the middle and pick the
lightest one. Thus it follows that our
algorithm is correct, since we compute the lightest triangle of the left half,
right half, and middle correctly, and take the minimum.
\newpage

%%%%%%%%%%%%%%%%%%%%%%%%%%%%%% Problem 3
\section*{Problem 5, CS38 Midterm, Matt Lim}
Let $x$ be some set vertex in $G$. We will let it be $v_k$ for some
$1 \le k \le n$. Let $HC(S,v)$ be 0 if there is no path from $x$ to $v$ that
passes through exactly the set $S$ and 1 if there is such a path. We will organize
our subsets of vertices in the table in the following way. $S_1, S_2, \dots,
S_n$ will be the subsets containing only a single vertex, with $S_1$ containing
$v_1$, $S_2$, containing $v_2$, etc. Next will come the subsets containing two
vertices, then three, and so on. This means $S_{2^n - 1}$ is the full set
of vertices.  Here is pseudo-code for our algorithm.

\vspace{5mm}
\noindent \textbf{Ham-cycle(G)}
\begin{enumerate}
    \item Test $G$ to see if it is a strong connected component. If yes,
        continue. If not, return false.
    \item for $i=1$ to $n$
    {\setlength\itemindent{25pt} \item for $S = S_1$ to $S_n$ }
    {\setlength\itemindent{50pt} \item if $v_k \in S$ }
    {\setlength\itemindent{75pt} \item $HC(S,v_i) = 1$ }
    {\setlength\itemindent{50pt} \item else }
    {\setlength\itemindent{75pt} \item $HC(S,v_i) = 0$ }
    \item for $i=1$ to $n$
    {\setlength\itemindent{25pt} \item for $S = S_{n+1}$ to $S_{2^n - 1}$ }
    {\setlength\itemindent{50pt} \item $HC(S,v_i) = 0$ }
    {\setlength\itemindent{50pt} \item if $v_i \in S$ }
    {\setlength\itemindent{75pt} \item for each edge $(u,v_i) \in E$ (if
        multiple such edges exist only consider one of them) }
    {\setlength\itemindent{100pt} \item if $v_i \neq v_k$ and $HC(S-v_i, u) == 1$ }
    {\setlength\itemindent{125pt} \item $HC(S, v_i) = 1$, break from loop }
    {\setlength\itemindent{100pt} \item elif $v_i == v_k$ and $HC(S, u) == 1$ }
    {\setlength\itemindent{125pt} \item $HC(S, v_i) = 1$, break from loop }
    \item if $HC(S_{2^n - 1}, v_k) == 1$
    {\setlength\itemindent{25pt} \item return $true$ }
    \item return $false$
\end{enumerate}

\vspace{5mm}
We will now show that this algorithm runs in time $O(n^2 2^n)$. Line 1 takes
$O(n+m)$ time, as seen in lecture 2. Lines 2-7 take time $O(n^2)$, since each
run through the loop is constant time. This leaves
lines 8-17. We can see that this builds a table of size order $n \cdot 2^n$. For
each cell of the table, we do order $n$ work, as we must look at all the
neighbors of the current vertex, of which their can be at most $n$. So overall,
these lines take $O(n^2 2^n)$ time, and we can see that the whole algorithm
takes this time as well.

We will now show correctness. First, note that we test the graph to see
if it is strongly connected. This is because if there is a Hamilton cycle in
$G$, then $G$ must be strongly connected. This is because if there is a
Hamilton cycle, every vertex is reachable from every other vertex through the
Hamilton cycle. Now, onto the rest of the algorithm. Note that each subproblem $(S,v)$ is to determine
if there is a path starting at $x$ that passes through exactly $S$ and ends at
$v$. The culmination of caching the results of all the subproblems is to
determine whether $HC(S_{2^n - 1},v_k)$ is 1, which would mean that there is a path
from $v_k$ to itself (a cycle) that passes through all the vertices. So, let us
explain how this table gets use the correct result for this cell. We have that
for a set of $n$ vertices, there are $2^n$ subsets of those vertices, including
the null set. This is why we have $2^n - 1$ columns (we don't include the null
set). For each subset $S$, we want to know if there is a path from $x$ to $v$ that
passes through exactly $S$ for every vertex $v$ in the graph (if we start and
end at $x$, then ``exactly $S$'' means that we start at $x$, pass through every
element in $S \backslash x$, and end at $x$). For lines 2-7, we only consider
subsets with one element. And since we start at vertex $x$, that means that the
only cell that will be filled with a one will be the one indexed by $(\{x\},
x)$. This is our initialization of the table. Next comes the dynamic programming
part. Let us examine lines 11-17. For each cell, we do the following. If $v_i
\notin S$, then the path cannot contain exactly $S$. So we can just mark the
cell with a 0. But if $v_i \in S$ and $v_i \neq v_k$, then we do the following. We consider all
vertices with edges that are directed towards $v_i$. That is, we consider all
vertices that can get to $v_i$ through a single edge. Then, for each of these
vertices, we consider if we can get to it from $x$, passing through $S-v_i$. If
we can, then we can clearly get to $v_i$ from $x$ passing through $S$. We do
something slightly different if $v_i \in S$ but $v_i == v_k$. We again consider
all vertices that can get to $v_i$ through a single edge. However, for each of
these vertices (we will call each one $u_k$),
we consider if we can get to $u_k$ from $x$ passing through $S$.
Notice that we do not alter $S$ here. This is because $v_i$ is our starting
vertex, and any paths from $x$ to $u_k$ must include $v_i$ itself. Thus we do
not want to subtract it out of $S$. Then it is clear that, if there is a path
from $x$ to some $u_k$ that passes through exactly $S$, then there is a path
from $x$ back to $v_i=x=v_k$ that passes through exactly $S$ as well (since here we
do not double count starting/finishing at $x$). So we have gone through all the
cases for each cell. Now we can see that, by building this table up cell by
cell, we get exactly the right answer for each subproblem. And eventually, all
we will have to do is look at each vertex that can get to $x$ using one edge and
see if any are reachable from $x$ going through $S_{2^n - 1}$. This will give us
the correct value for $HC(S_{2^n - 1}, x)$ and thus the correct answer to our
problem. Thus we can conclude that our algorithm works correctly.
\newpage
\end{document}
