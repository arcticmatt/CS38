%%%%%%%%%%%%%%%%%%%%%%%%%%%%%% Preamble
\documentclass{article}
\usepackage{amsmath,amssymb,amsthm,fullpage,enumitem}
\usepackage[a4paper,bindingoffset=0in,left=1in,right=1in,top=1in,
bottom=1in,footskip=0in]{geometry}
\newtheorem*{prop}{Proposition}
%\newcounter{Examplecount}
%\setcounter{Examplecount}{0}
\newenvironment{discussion}{\noindent Discussion.}{}
\pagenumbering{gobble}
\newcommand{\modifyenum}[1]{%
  \end{enumerate}
  \begin{enumerate}[resume,#1]
}
\begin{document}

%%%%%%%%%%%%%%%%%%%%%%%%%%%%%% Problem 1
\section*{Problem 1, CS38 Final, Matt Lim}
The greedy algorithm will be as follows. At each step, we will pick the set
covering the largest number of remaining uncovered items. Now, we will show that
this achieves an approximation ratio of $\frac{e}{e-1}$.

Let $OPT$ be the cardinality of the set of elements covered by a
\textit{maximum k-cover} - that is, the cardinality of the set
with the maximum number of elements of $U$ that can be covered by $k$ of the
subsets. Let $r_i$ be the number of the $OPT$ elements remaining (not yet
covered) after iteration $i$; that is, if $x$ elements have been covered after
iteration $j$, then $r_j = OPT - x$. This means $r_0 = OPT$. Then we claim that
\[ r_i \leq (1 - \frac{1}{k}) \cdot r_{i-1}. \]
The proof for this statement is as follows. We have
that $k$ subsets cover $OPT$ elements. Then, at each step, $k$ subsets cover
all remaining elements (out of the $OPT$ elements, not the universe).
Thus, at each step, \textit{some} subset must cover at least a $\frac{1}{k}$
fraction of those remaining elements out of the $OPT$ elements. That is, at
each step, there
must exist some subset that covers a fraction $\frac{1}{k}$ (at least) of the $r_{i-1}$
elements remaining to be covered. This basically means we can cover at least
$\frac{r_{i-1}}{k}$ uncovered elements at each step $i$. And since our
greedy algorithm picks the set covering the largest number of remaining
uncovered items, we will cover at least $\frac{r_{i-1}}{k}$ uncovered remaining
elements at each step $i$. Then, using this claim, we have that
\[ r_i \leq (1 - \frac{1}{k})^i \cdot OPT \]
\[ r_k \leq (1 - \frac{1}{k})^k \cdot OPT \]
\[ r_k \leq \frac{OPT}{e} \]
So we have that, after $k$ iterations of our algorithm (which means we have
picked $k$ sets) there are less than or equal to $\frac{OPT}{e}$ elements
remaining of the $OPT$ number of elements that are possible to be covered with
$k$ subsets. This means that $c$, the number of elements we have
covered out of the $OPT$ possible, is bounded below as follows:
\[ c \geq OPT - \frac{OPT}{e} \]
\[ c \geq OPT(1 - \frac{1}{e}) \]
\[ c \geq OPT(\frac{e-1}{e}) \]
\[ c(\frac{e}{e-1}) \geq OPT \]
Thus we get our approximation ratio of $\frac{e}{e-1}$.
\newpage

%%%%%%%%%%%%%%%%%%%%%%%%%%%%%% Problem 2
\section*{Problem 2, CS38 Final, Matt Lim}
Here is our algorithm.

\vspace{5mm}

\textbf{Maximum-matching-tree}$(\text{tree } G = (V,E))$
\begin{enumerate}
    \item Root the tree $G$ at a node $r$
    \item $S = \varnothing$
    \item $OPT_{in}[v] = OPT_{out}[v] = 0$ for all $v \in V$
    \item \textbf{foreach} node $u$ of $G$ in postorder
    \modifyenum{leftmargin=50pt}
    \item \textbf{if} $u$ is a leaf
    \modifyenum{leftmargin=75pt}
    \item $OPT_{in}[u] = OPT_{out}[u] = 0$
    \modifyenum{leftmargin=50pt}
    \item \textbf{else}
    \modifyenum{leftmargin=75pt}
    \item \textbf{foreach} child edge $(u,w)$
    \modifyenum{leftmargin=100pt}
    \item int $new = 1 + OPT_{out}[w] +
        \sum_{v \in \text{children($u$), $v \neq w$}} max\{OPT_{out}[v],
        OPT_{in}[v]\}$
    \modifyenum{leftmargin=100pt}
    \item \textbf{if} $new > OPT_{in}[u]$
    \modifyenum{leftmargin=125pt}
    \item $OTP_{in}[u] = new$
    \modifyenum{leftmargin=125pt}
    \item Store the following information in
        $B_{in}[u]$: edge $e = (u,w)$; for all children $v$ of $u$ (except $w$),
        if $OPT_{out}[v] \geq OPT_{in}[v]$ store $B_{out}[v]$ and
        if $OPT_{out}[v] < OPT_{in}[v]$ store $B_{in}[v]$; for $w$, store
        $B_{out}[w]$
    \modifyenum{leftmargin=75pt}
    \item $OPT_{out}[u] =
        \sum_{v \in \text{children($u$)}} max\{OPT_{in}[v], OPT_{out}[v]$\}
    \modifyenum{leftmargin=75pt}
    \item Store the following information in
        $B_{out}[u]$: for all children $v$ of $u$,
        if $OPT_{out}[v] \geq OPT_{in}[v]$ store $B_{out}[v]$ and
        if $OPT_{out}[v] < OPT_{in}[v]$ store $B_{in}[v]$
    \modifyenum{leftmargin=25pt}
    \item \textbf{if} $OPT_{in}[r] > OPT_{out}[r]$
    \modifyenum{leftmargin=50pt}
    \item Backtrack starting with $B_{in}[r]$; for
        each $B_{in}[x]$ encountered, add the stored edge $e$ to $S$
    \modifyenum{leftmargin=25pt}
    \item \textbf{else}
    \modifyenum{leftmargin=50pt}
    \item Backtrack starting with $B_{out}[r]$; for each $B_{in}[x]$ encountered,
        add the stored edge $e$ to $S$
\end{enumerate}

\vspace{5mm}
We will now show that our algorithm runs in time $O(|E|)$ operations, where an
operation is as defined in the problem. First we iterate through all the leaves,
of which their are order $|E|$ many of, and set some values. This is clearly
within the time bounds. Then, we can see the combination of lines 4
and 7 runs through every edge in the graph. So, for every edge, we
calculate a sum (line 9); then if the value of this sum is greater than the
current value of $OPT_{in}[u]$, we set some variables. According to Lecture 17
slide 18 and our definition of an operation, we can can treat the sum in line 9
as one operation. Then, since line 12 just takes information from that sum and
stores it, line
12 is also just one operation. Then for each non-leaf node (order $|E|$ many),
we also calculate another sum
in line 13, and set another variable in line 14. For the same reasons as lines 9
and 12, these two lines are also just one operation each. Then, we have that
there are a constant number of operations for each edge. Thus this algorithm
runs in time $O(|E|)$ operations.

Now we will show that our algorithm works. Note that this depends on the fact
that $G$ is a tree. We can see that we visit the nodes in
postorder. This is to ensure a node is visited after all its children. For each
node $u$, we consider what happens, each child edge $e$ at a time, if we add
$e$ to the matching and what happens if we don't add any of its child
edges to the matching. The former part occurs in lines 8-11. In line 9, we
consider adding an edge $(u,w)$ to the matching. This would mean that our matching
cardinality would go up 1, that $w$ could not have any of its child edges in the
matching, and the other children of $u$ (not $w$) can have one of their child
edges in the matching or not. After calculating each such possibility (one
possibility for each child edge $(u,w)$) we make it the new value of
$OPT_{in}[u]$ if it is greater than the current value. Doing this in the
for-loop ensures we get the maximum score for including one child edge of $u$.
The latter part occurs in line 11, where we calculate the maximum score of when
we don't include any of $u$'s child edges in the matching. In this way, we
compute the value of a maximum matching for the subtrees rooted at each node (it
will be the maximum of $OPT_{in}$ and $OPT_{out}$).
Thus, we compute the value of a maximum matching for a tree rooted at $r$, which
is just the value of the maximum matching for the whole tree. Now we must
explain why our backtracking works. We have that, for each node $u$, we store
$B_{in}[u]$ and $B_{out}[u]$. $B_{in}[u]$ contains the optimal child edge
$(u,w)$ of $u$ to add to the matching, and the optimal in/out choice for all child vertices of
$u$ (for $w$, the choice is by default ``out'') stored as $B$ values. $B_{out}[u]$ contains the
optimal in/out choice for all child vertices of $u$ stored as $B$ values. Thus, when starting from the
root node $r$, we have the optimal choices for all its child vertices contained
in $B_{in}[r]$ and $B_{out}[r]$. Then, we know which one to start with from the values
of $OPT_{in}[r]$ and $OPT_{out}[r]$. This continues down to the bottom of the
tree; at each level, from all the $B$ values, we know the optimal choices/$B$
values for the next level. And since we add the $e$ values stored by $B_{in}$'s,
which we know are optimal, we get the maximum matching.
\newpage

%%%%%%%%%%%%%%%%%%%%%%%%%%%%%% Problem 3
\section*{Problem 3, CS38 Final, Matt Lim}
Here is the pseudo-code of our algorithm to find a maximum sum subsequence. Note
that it does not use divide-and-conquer; \textbf{I asked Professor Umans and he said it
is OK as long as it runs in the desired time bounds.} Let the size of the
list be $n$, and let $L_k$ denote the $k$th element of the list.
Note that we will let the ``value'' of a list be the sum of its elements.

\vspace{5mm}

\textbf{Maximum-sum-subsequence}(List $L$)
\begin{enumerate}
    \item check if $L$ is all non-positive elements by iterating through every
        element. While iterating through, keep track of the maximum element's index.
        If $L$ is all non-positive elements, return index of maximum element.
    \item int $curr = max = 0$
    \item int $begin = end = beginMax = endMax = 1$
    \item \textbf{for} $1 \leq k \leq n$
    {\setlength\itemindent{25pt} \item \textbf{if} $curr + L_k > max$ }
    {\setlength\itemindent{50pt} \item $max = curr + L_k$ }
    {\setlength\itemindent{50pt} \item $curr = curr + L_k$ }
    {\setlength\itemindent{50pt} \item $beginMax = begin$ }
    {\setlength\itemindent{50pt} \item $endMax = k$ }
    {\setlength\itemindent{25pt} \item \textbf{elif} $curr + L_k > 0$ }
    {\setlength\itemindent{50pt} \item $curr = curr + L_k$ }
    {\setlength\itemindent{25pt} \item \textbf{else} }
    {\setlength\itemindent{50pt} \item $curr = 0$ }
    {\setlength\itemindent{50pt} \item $begin = end = k+1$ }
    \item return $beginMax, endMax$
\end{enumerate}

\vspace{5mm}
We will now see why this runs in the desired time bounds. We have that step 1
takes $O(n)$ operations, as it just iterates through the list. Then, the rest of
the algorithm is basically just a for-loop that iterates through all $n$
elements of the list. In the for-loop, we are just adding integers of
magnitude $O(n|\text{max$_i a_i$}|)$, comparing values, and/or setting integer values.
And in each iteration of the loop, the number of these operations that happen is
clearly bounded by a constant. So we have that overall, our algorithm only takes
$O(n)$ operations, which meets the desired bound of $O(n\log n)$ operations.

Now we will explain why this algorithm works. The first line takes care of lists
that have no positive values by simply returning the maximum value. Otherwise,
we do the following. We start from the
beginning and iterate through the whole list. We keep track of the starting and
ending indices of the current subsequence ($begin,end$), the starting and ending
indices of the current maximum subsequence ($beginMax,endMax$), and the values
for the current subsequence and the maximum subsequence ($curr,max$). For every element, we add its
value to $curr$. Let the element we are considering be called $L_k$.
So, we will add that element to $curr$ to get $sum = curr + L_k$. If $sum$ is
greater than $max$, we set the current subsequence to be the maximum. If it is
not greater than $max$ but greater than $0$, we maintain our current
subsequence. Else, if $sum$ is less than or equal to $0$, we start a new
subsequence on the next element. We do this because once the value of a current
subsequence is non-positive, we know that it is better or the same to start a new
subsequence on the next element than to continue on with the current one, as
non-positive values do not help the sum. So overall, our algorithm considers
a subsequence until its value becomes non-positive (keeping track of the
maximum along the way), and when its value does become non-positive, it
considers a new subsequence starting on the next element. Thus, our algorithm
only elongates the current subsequence if it is possible that it is an essential
part of the maximum (that it actually adds to the value), all while keeping track of
the current maximum subsequence
- otherwise, it starts a new one, since extending the previous one cannot help.
In other words, we never make an outright bad move, only moves that are or have
the possibility of being beneficial. And if we make a possibly beneficial move
that hurts the current value, like adding a negative to our subsequence, we have
stored the value of the max before that move.
So clearly, our algorithm will not miss finding a maximum sum subsequence.
\newpage

%%%%%%%%%%%%%%%%%%%%%%%%%%%%%% Problem 3
\section*{Problem 4, CS38 Final, Matt Lim}
\begin{description}
    \item[(a)] Given the LP, our constraint matrix $A$ will be as follows. Letting
        $|V| = n$ and $|E| = m$, we have that our constraint matrix $A$ will
        have $m$ rows and $m + n$ columns. The $m$ rows will represent the $m$
        edges. The first $m$ rows will also represent the $m$ edges, and the
        last $n$ rows will represent all the vertices. Looking at a
        row that represents edge $e_i = (u,v)$, there will be a $1$ in the
        column that also represents edge $e_i$. Also, there will be a $-1$ in
        the column that represents vertex $u$, and a $1$ in the column that
        represents vertex $v$. All other columns in that row will have a $0$.
        Thus, with this constraint matrix in place, we
        can see that multiplying it by the column vector $x$ with the edge
        variables on top (same order as the edges in the constraint matrix) and
        the vertex variables on the bottom (same order as the vertices in the
        constraint matrix) and constraining the product to be greater than or
        equal to a column vector of $0$s, we meet our desired constraint for
        each edge. We will now prove that $A$ is unimodular.

        Now we will prove by induction that $A$ is unimodular. To do this we
        will prove that every square submatrix of $A$ has determinant $0$, $1$,
        or $-1$.
        First we will consider the base case. So, consider any submatrix of $A$
        that has dimensions $1 \times 1$. We showed above that every cell of $A$
        is either $0$, $1$, or $-1$. And since the determinant of a $1 \times 1$
        matrix is just the value of the cell, we have that every $1 \times 1$
        submatrix of $A$ has determinant $0$, $1$, or $-1$. Now we will do our inductive
        assumption. So, assume that every $k \times k$ submatrix of
        $A$, where $1 \le k \le i - 1$ has determinant $0$, $1$, or $-1$.
        Now we must show that every $i \times i$ submatrix of
        $A$ has such a determinant. So consider an arbitrary $i \times i$
        submatrix $B$ of $A$.

        Now we will consider two cases. The first case is when a column $c$ in
        $B$ is made of all
        zeroes. Then we have that the determinant of $B$ is zero (expand by
        cofactors along $c$). The second case is
        when a column $c$ in $B$ has one $1$ or one $-1$ and $0$s everywhere else.
        Here we have that the determinant of $B$ is either $0$, $1$,
        or $-1$. We can see this is true if we expand by cofactors along $c$.
        This is because by calculating the determinant this way, it is the same
        as if we calculate the determinant of the $(i - 1) \times (i - 1)$
        matrix we get by crossing out column $c$ and the row with the one $1/-1$ in
        column $c$ and multiplying it by $1$ or $-1$. And by our inductive
        assumption, the determinant of such a $(i - 1) \times (i - 1)$ matrix is
        $0$, $1$, or $-1$. So clearly the determinant of $B$ is $0$, $1$, or
        $-1$. Note that these first two cases cover every square submatrix that has a
        column from the first $m$ columns of $A$, since all those columns only
        have one $1$ (since we don't include the same edge twice). Thus, when we
        take a submatrix that includes one of those columns, the corresponding
        column in the submatrix will have at most one $1$.

        The next cases cover submatrices that only take columns from the last
        $n$ columns of $A$, since, as we noted above, the first two cases cover
        all cases when a column from the first $m$ columns of $A$ is included.
        So, we get the following cases. Similarly to the first two cases, we
        have a third case in which one row of $B$ is made of all $0$s and a
        fourth case in which one row of $B$ is made of all $0$s and one
        $1$ or $-1$. These cases are proved true in the same way as the first
        two cases, except that instead of expanding along a column we expand
        along a row. Then, we have a final fifth case, in which every row of $B$ has
        both one $1$ and one $-1$ and every column has at least 2 non-zero
        values. Note that we never get a case where there are
        two $1$s and one $-1$ in a row, because that would involve including a
        column from the first $m$ columns (a case which we already took care of).
        We will now claim that in this case, it is possible to make a row of
        all zeroes by adding multiples of rows to other rows. We
        can see this is true because of the following reason. Since we have two
        non-zero values in each column (two $1$s, two $-1$s, or
        one $1$ and one $-1$), we can do the following. Let us add rows to the
        first row. Then, we can just keep on adding rows, getting rid of one
        $\pm1$ in a column one column at a time. The first row will then eventually become all $0$s,
        since we have two non-zero values in each column that we can have cancel
        and we only have two $\pm1$s in every row.
        For an example, see the following:
        \[
        \begin{bmatrix}
        -1 & 1 & 0 & 0\\
        -1 & 0 & 1 & 0\\
         0 & -1 & 0 & 1\\
         0 & 0 & -1 & 1
        \end{bmatrix} \rightarrow
        \begin{bmatrix}
         0 & 1 & -1 & 0\\
        -1 & 0 & 1 & 0\\
         0 & -1 & 0 & 1\\
         0 & 0 & -1 & 1
        \end{bmatrix} \rightarrow
        \begin{bmatrix}
         0 & 0 & -1 & 1\\
        -1 & 0 & 1 & 0\\
         0 & -1 & 0 & 1\\
         0 & 0 & -1 & 1
        \end{bmatrix} \rightarrow
        \begin{bmatrix}
         0 & 0 & 0 & 0\\
        -1 & 0 & 1 & 0\\
         0 & -1 & 0 & 1\\
         0 & 0 & -1 & 1
        \end{bmatrix}
        \]
        First we subtract the second row, then we add the third row, then we
        subtract the fourth row. We can see that, since we have two $\pm1$s in
        each column, we can just cancel all of them out in the first row.
        So, we have that it is possible to make a row of all zeroes by adding
        multiples of rows to other rows. Note now that it is a theorem of linear
        algebra that the determinant of a square matrix is unchanged if the
        entries in one row are added to those in another row (page 1224 CLRS).
        Then, since we have a row of all zeroes, we can expand along this row
        and get that the determinant of $B$ is $0$. So for this case, the determinant
        is $0$. Now we have that all the cases are covered. Thus, we can
        conclude by induction that every square submatrix of $A$ has determinant
        $0$, $1$, or $-1$, and thus that $A$ is totally unimodular.

        \vspace{5mm}
        \textbf{TODO} \\
        We will do this by induction. So, consider the bases case of
        $2 \times 2$ matrices. We have four possibilities, as follows:
        i\[ \begin{bmatrix}
        -1 & 1\\
        -1 & 1
        \end{bmatrix}
        \begin{bmatrix}
        1 & -1\\
        1 & -1
        \end{bmatrix}
        \begin{bmatrix}
        1 & -1\\
        -1 & 1
        \end{bmatrix}
        \begin{bmatrix}
        -1 & 1\\
        1 & -1
        \end{bmatrix}
        \]
        Clearly, in each of these possibilities, we can make a row of all
        zeroes. In the first two, we can do this by subtracting the second row
        from the first row, and in the last two we can do this by adding the
        second row to the first row. So, the base case has been proved. Now for
        the inductive assumption. So, assume that for every $k \times k$ submatrix of
        $A$ with columns only from the last $n$ columns and with both a $1$ and
        a $-1$ in each row, where $1 \le k \le i - 1$, it is possible to make a
        row of all zeroes by adding multiples of rows to other rows.
        Now we must show that this is true for every $i \times i$ submatrix of
        $A$  with columns only from the last $n$ columns and with both a $1$ and
        a $-1$ in each row. So, let us consider an arbitrary $i-1 \times i-1$
        submatrix of $A$ with columns only from the last $n$ columns and with
        both a $1$ and a $-1$ in each row. Now we will consider adding a column
        and a row to this submatrix to make an $i \times i$ submatrix with
        columns only from the last $n$ columns and with both a $1$ and a $-1$ in each row.
        To do this, the column we add can only have a $1$ or $-1$ in the new row
        we add. So basically, we add an $i-1$ length column with all $0$s, then
        an arbitrary row with one $1$ and one $-1$.
    \item[(b)] We are given that this LP is the dual of the max-flow LP. Thus,
        the minimum/optimum value given by this LP equals the maximum value
        given by the max-flow LP. This is by the strong duality theorem given in
        lecture. We can apply this because, clearly, $A$ and $b$ are
        real-valued. $c$ is also clearly real valued, as we have integer
        capacities for each edge. And clearly the dual and primal are nonempty.
        Then we have that the value of this LP equals the value of the max-flow.
        By the max-flow min-cut theorem, the value of the max-flow equals the
        capacity of the min-cut. Thus we have that the value of this LP equals
        the value of the min-cut.

\end{description}
\newpage

%%%%%%%%%%%%%%%%%%%%%%%%%%%%%% Problem 3
\end{document}
